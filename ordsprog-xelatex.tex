% -*- TeX-engine: xetex coding: utf-8 -*-
\documentclass[a4paper]{article}
\usepackage[para]{footmisc}
\usepackage{fullpage}
\usepackage{fontspec}
%\usepackage{paralist}
\usepackage{tabu}
\usepackage{hyperref}
%\usepackage{navigator}
%\embeddedfile{sourcecode}{./\jobname.tex}
\setromanfont[Ligatures=TeX]{CMU Serif}
\setsansfont[Ligatures=TeX]{CMU Sans Serif}
\setmonofont[Ligatures=TeX]{CMU Typewriter Text}
% \setmainfont[Ligatures=TeX]{Linux Libertine O}
\usepackage{polyglossia}
\usepackage{marginnote}
\setdefaultlanguage[]{danish}
\setotherlanguage[]{russian}
\pagestyle{headings}
\usepackage{multind}
\usepackage{xunicode}
%\usepackage{pfnote}https://cloud.sagemath.com/dd54c4e6-02d7-4035-8bf0-adf80f8e0bef/raw/.smc/root/tmp/.9ad549db-2b69-4e1a-a07b-ab95bdf27726/1.png
%\usepackage{dblfnote}
%\DFNalwaysdouble % for this example
\usepackage{perpage} %the perpage package
\MakePerPage{footnote} %the perpage package command

\title{Danske ordsprog --- med russiske paralleller}
\author{Knud-Erik Kjær Madsen \and Redigeret af Sigurd Meldgaard}
\date{}
\makeindex{russisk}
\makeindex{dansk}
\begin{document}
\maketitle
\newenvironment{overslist}
{}{}
%{\begin{compactitem}}
%{\end{compactitem}}
\newcommand{\mitem}{\par\hangindent=0.35cm\textbullet\ }
%\par
%\noindent
%\hskip -.25cm \raisebox{.02cm}{$\bullet$}\hskip 3pt}%\item}
\newcommand{\ordsprog}[2]{
\noindent
\begin{tabu}to \textwidth {XX[2]}
\textdanish{#1}
  &
\begin{overslist}%
\hangindent=0.35cm
\textbullet\
\begin{russian}#2\par
\end{russian}
\end{overslist}%
\end{tabu}
\vskip5mm
}
\newcommand{\ru}[1]{\begin{russian}#1\end{russian}}
\newenvironment{letter}[1]
{\vskip0.5cm \section*{#1}
}
{}
\newcommand{\dui}[1]{\index{dansk}{#1}}
\newcommand{\di}[1]{\index{dansk}{#1}#1}
\newcommand{\did}[2]{\index{dansk}{#1}#2}
\newcommand{\mw}[1]{\index{dansk}{#1}\emph{#1}}
\newcommand{\mwni}[1]{\emph{#1}}
\newcommand{\ri}[1]{\index{russisk}{#1}#1}
\newcommand{\rid}[2]{\index{russisk}{#1}#2}
\newcommand{\fn}[1]{}
%\newcommand{\fn}[1]{\footnote{\textdanish{#1}}}
Ordsprogene er udvalgt så de har klare russiske paralleller og er ordnet alfabetisk efter deres danske emne (fremhævet med kursiv).

Stor tak til Annie Christensen og Svetlana Alekceebha Chouvalova for tålmodigt at stille deres faglige ekspertise til rådighed og komme med utallige forslag til rettelser og forbedringer.
%%%%%%%%%%%%%%%%%%%%%%%%%%%%%%%
% -*- TeX-engine: xetex, coding: utf-8 -*-
% ;;  1! 2ё 3ъ 4Ё 5 6^ 7& 8* 9( 0) -_ Ч  Ю
% ;;   Я  В  Е  Р  Т  Ы  У  И  О  П  Ш  Щ
% ;;    А  С  Д  Ф  Г  Х  Й  К  Л  ;: '" Э
% ;;     З  Ь  Ц  Ж  Б  Н  М  ,< .> /?

% ;;  1! 2ё 3ъ 4Ё 5% 6^ 7& 8* 9( 0) -_ Ч  Ю
% ;;   Ђ  Ѓ  Є  Ѕ  І  Ї  Ј  Љ  Њ  Ћ  Ш  Щ
% ;;    Ќ  Ў  Џ  Ф  Г  Х  Й  К  Л  ;: '" Э
% ;;     З  Ь  Ц  Ж  Б  Н  М  ,< .> /?
\letter{A}
\dui{a}
\ordsprog{Har man sagt \mwni{A}, må man også sige \di{B}.}
         {Кто сказа́л \ri{А}, тот до́лжен сказа́ть и \ri{Б}.
%          \mitem Взя́лся за гуж\fn{bestemt rem på seletøj der holder forbinder stangen til seletøjet}, не говори́, что не дюж\fn{stærk}.
         }
\ordsprog{Aber forbliver \mw{aber},
               selv om de \mbox{klæder} sig i \di{fløjl} og \di{silke}.}
        {Свинья́ и в золото́м оше́йнике\fn{halsbånd} --- \mbox{\ri{свинья́}}.}
\dui{adel}
\ordsprog{\mwni{Adel} for\di{pligt}er.}
        {\rid{положе́ние}{Положе́ние}\fn{stilling} обя́зывает.}
\dui{alderdom}
\ordsprog{\mwni{Alderdom} er knarvorn.
%(Alderdommens byr\-de er den tungeste af alle.)
}
        {\rid{ста́рость}{Ста́рость} не \ri{ра́дость}.
         \mitem  Ста́рость не кра́сные\fn{smukke, skønne} дни.}
\dui{alderdom}
\ordsprog{\mwni{Alderdom} og \di{visdom} følges ikke altid ad.}
         {До́лго про́жил, да \ri{ум}а́ не на́жил\fn{\ru{на́жить}=erhverve}.
        \mitem Борода́ вы́росла, а ума́ не вы́несла\fn{\ru{вы́нести}=medbringe}.
        \mitem Мо́лодость не без глу́пости, ста́рость не без ду́рости.}
\dui{alderdom}
% \ordsprog{To \mw{alen} af et stykke.}
%         {Они́ два \rid{сапо́г}{сапога́}
%         па́ра.
%         \mitem Они́ одного́
%         \rid{по́ле}{по́ля}\fn{mark} я́года\fn{bær}.}
% \ordsprog{Én for \mw{alle} og alle for én.}
%         {\rid{оди́н}{Оди́н} за всех, \ri{все} за одного́.}
\dui{ansigtet}
\dui{øje}
\ordsprog{\mwni{Ansigtet}(øjnene) er sjælens spejl.}
        {
%         на \rid{лицо́}{лице́} не \mbox{утаи́тся}
%         Глава́ зе́ркало души́.
%         \mitem Что в се́рдце ва́рится\fn{\ru{ва́риться}=brygge},
        Глаза́ --- зе́рколо души́.
        }
\dui{appetit}
\ordsprog{\mwni{Appetitten} kommer med maden.}
         {\rid{аппети́т}{Аппети́т} прихо́дит во
            вре́мя \rid{еда́}{еды́}.}
\dui{april}
\ordsprog{\mwni{Aprils} væde er bondens glæde.}
         {Апре́ль с водо́ю, май с траво́ю\fn{græs}.}
\ordsprog{Den, der ikke vil \mw{arbejde}, skal heller ikke have \di{føde}n.}
        {Кто не рабо́тает, тот и не ест.
        \mitem \rid{рабо́тать}{Рабо́тать} не
        заста́вят\fn{tvinges}, и есть не \mbox{поса́дят}\fn{\ru{посади́ть}=bringe til sæde}.}
\ordsprog{Lidet \mw{arbejde}, liden \di{løn}.}
        {\rid{По заслу́га}{По заслу́гам}\fn{tjeneste} и награ́да\fn{løn}.}
\ordsprog{Hvor der er \mw{arbejde}, er der lykke.}
         {Где \ri{труд}, там и \ri{сча́стье}.}
\dui{arbejder}
\ordsprog{\mwni{Arbejderen} er sin \di{løn} værd.}
         {Како́в \ri{рабо́тник}, такова́ ему́ и \ri{пла́та}.
        \mitem Какова́ \ri{рабо́та}, такова́  и \ri{пла́та}.
        \mitem По рабо́́те пла́та.}
\ordsprog{Den ene har \mw{arbejdet}, den anden lønnen.}
        {\rid{медве́д}{Медве́дь}\fn{bjørn} пля́шет\fn{\ru{пля́сать}=danse}, а цыга́н\fn{sigøjner} \ri{де́ньги} берёт.}
%         \mitem Оди́н собира́ет\fn{gaber}, друго́й зева́ет\fn{gaber}.
\ordsprog{\mwni{Arbejdet} bærer lønnen i sig selv.}
{Рабо́та уже́ само́ по себе́ явля́ется\fn{viser sig} награ́дой.}
\letter{B}
\dui{bage}
\ordsprog{Der \mwni{bages} også godt \di{brød} i \di{fremmede lande}.}
        {\rid{хлеб}{Хлеб} везде́\fn{overalt} хоро́ш, и у нас, и за́ морем.}
\ordsprog{Man skal ikke give \mw{bagerbørn} hvedebrød.}
        {В Ту́лу\fn{byen Tula er kendt for sine samovarer.} со свои́м самова́ром не е́здят.
        \mitem В лес дрова́\fn{brænde} не во́зят, в \ri{коло́дец}\fn{brønd} воды́ не льют\fn{hælder}.}
\ordsprog{Hellere \di{forklog} end \mw{bagklog}.}
        {Челове́к за́дним умо́м кре́пок\fn{stærk}.
        \mitem За́дним умо́м\fn{bagklogskab} де́ла не попра́вишь\fn{retter op}.
        \mitem Пора́ ушла́, так ум пришёл.}
\ordsprog{Brændt \mw{barn} skyr ilden.}
        {Ожёгся\fn{brændt} на молоке́, ста́нешь дуть\fn{blæse} и на́ воду.
        \mitem Обжёгшись на молоке́, бу́дешь дуть и на́ воду.
        \mitem Пу́ганая\fn{skræmt} воро́на\fn{krage} куста́\fn{busk} бои́тся.}
\ordsprog{Kært \mw{barn} har mange \di{navne}.}
        {У ми́лого дитя́ти мно́го имён.}
\dui{barn}
\ordsprog{Det er for sent at kaste \di{brønden} til, når \mwni{barnet}
er druknet.}
        {Парни́шка\fn{knøs} ввали́лся\fn{faldt}, так и\fn{\ru{так и}=såvel som} коло́дец накры́ли.
        \mitem По́сле дра́ки\fn{slåskamp} кулака́ми\fn{næverne} не ма́шут\fn{knytter}.}
\dui{begynde}
\ordsprog{Godt \mwni{begyndt} er halvt \di{fuldendt}.}
        {Хоро́шее нача́ло --- полови́на де́ла.
        \mitem До́брое \ri{нача́ло} не без \rid{коне́ц}{конца́}.}
\ordsprog{Al \mw{begyndelse} er svær.}
        {         Пе́рвый \ri{блин} --- ко́мом\fn{klump}.
        \mitem Стра́шно де́ло до начи́ну.
\mitem Лиха́\fn{\ru{лихо́й}=slem, svær} беда́\fn{ulykke}  --- нача́ло.}
\dui{ben}
\ordsprog{Den, der kommer sent, må gnave \mwni{benene}.}
        {По́здно пришёл, то́лько \rid{кость}{ко́сти}\fn{knogler} нашёл.
        \mitem Кто по́здно пришёл, тому́ мосо́л\fn{kødben}.}
\dui{bjerg}
\ordsprog{\mwni{Bjerget} skal barsle, og der fødes en latterlig mus.}
        {\ri{Гора́} родила́ \ri{мышь}.}
\dui{bjerg}
\ordsprog{Når \mwni{bjerget} ikke vil komme til Muhamed, må Muhamed
komme til bjerget.}
        {Е́сли \ri{гора́} не идёт к \rid{Магоме́т}{Магоме́ту}, то
        Магоме́т идёт к горе́.}
\dui{bjerg}
\ordsprog{Tro flytter bjerge.}
         {Ве́ра дви́гает го́ры.}
\dui{bjørn}
\ordsprog{Man skal ikke sælge skindet, før \mwni{bjørnen} er skudt.}
        {Не уби́в \rid{медве́д}{медве́дя}, шку́ры\fn{huden} не продава́й\fn{sælg}.
        \mitem Не дели́ шку́ру неуби́того медве́дя.}
%\ordsprog{Det, man først bliver \mw{blind} på, er øjnene.}
%        {Не \ri{ви́деть}, что под но́сом твори́тся (и где
%        то́лько бы́ли мои́ \ri{глаза́}).}
\dui{blind}
\ordsprog{I de \mwni{blindes} rige er den enøjede konge.}
        {В слепо́м ца́рстве \ri{криво́й}\fn{enøjet}  --- \ri{царь}.
        \mitem На безры́бье\fn{fiskemangel} и рак\fn{krebs} \ri{ры́ба}.
        \mitem Меж слепы́х и криво́й зря́чий\fn{seende}.}
\dui{blod}
\ordsprog{\mwni{Blodet} er aldrig så tyndt, det er jo tykkere end vand.}
        {Кро́вное\fn{blod-} родство́\fn{afstamning} пре́жде всего́.}
% \ordsprog{Den, der sår \mw{blæst}, skal høste \mbox{storm}.}
%         {Кто посе́ет ве́тер, пожнёт бу́рю.
%         \mitem Посе́явший ве́тер пожнёт\fn{høste} бу́рю\fn{storm}.}
\ordsprog{Man skal ikke slå større \mw{brød} op, end man kan bage.}
        {
        Взя́ться за непоси́льное де́ло.
%         Кто за всё хвата́ется\fn{griber om}, тот ничего́ не сде́лает\fn{udfører}.
%         \mitem За всё бра́ться, ничего́ не сде́лать.
% %        \mitem Не бери́сь за гуж\fn{del af seletøj}, ко́ли\fn{hvis} не дюж.
%         \mitem Взя́вшись за де́ло, не говори́ что слаб\fn{svag},
%                 что не хвата́ет\fn{slår til} сил.
%        % slettet \mitem Мно́гого жела́ть, добра́ не вида́ть.
%         \mitem Мно́гого иска́ть\fn{søge} ста́нешь\fn{\ru{стать}=give sig til}, ничего́ не доста́нешь\fn{ru{доста́ть}=opnå}.
%         \mitem Мно́гого жела́л, да ничего́ не пойма́л\fn{fik fat på}.
        }
\ordsprog{Hårdt \mw{brød} gør kinden rød.}
        {\rid{хлеб}{Хлеб} да\fn{og} \ri{вода́} --- здоро́вая еда́.}
\ordsprog{Den enes \mw{brød}, den andens \di{død}.}
        {И то сча́стье, что ино́му вёдро\fn{godt vejr}, ино́му
        несча́стье.
        \mitem \rid{стро́ить}{Стро́ить}\fn{at bygge} своё \ri{сча́стье} на
        чужо́м\fn{fremmeds} \ri{несча́стье}.
        \mitem \rid{Ко́шка}{Ко́шке} игру́шки\fn{legetøj}, а мы́шке слёзки\fn{tårer}.}
\dui{bukser}
\ordsprog{Man skal ikke \di{skræve} længere, end \mwni{bukserne} kan
        holde.}
        {По одёжке\fn{klæder} протя́гивай\fn{stræk} но́жки\fn{benene}.}
\dui{busk}
\ordsprog{Den, der er ræd for \mwni{busken}, kommer sent til skovs.}
        {Во́лка\fn{ulv} боя́ться, в лес не ходи́ть.
        \mitem Сме́рти\fn{døden} боя́ться, на све́те\fn{verden} не жить.}
\dui{bæk}
\ordsprog{Mange \mwni{bække} små gør en stor \di{å}.}
        {С ми́ру по ни́тке\fn{tråd} --- го́лому' руба́шка\fn{undertrøje}.
        \mitem Из ма́лого скла́дывается\fn{skabes} вели́кое.
        \mitem Из кро́шек\fn{krummer} ку́чкa\fn{håndfuld}, из ка́пель\fn{dråber} мо́ре\fn{hav}.
        \mitem По ни́тке и до клубка́\fn{garnnøgle} дохо́дят\fn{når}.}
\ordsprog{Af \mw{børn} og \di{fulde folk} skal man høre
        \di{sandheden}.}
        {Уста́ми\fn{mund} младе́нца\fn{barn} (младе́нцев) глаго́лет\fn{lyder} и́стина\fn{sandhed}.
        \mitem Глу́пый\fn{dumme} да ма́лый всегда́
                и́стину (пра́вду) говоря́т.}
\ordsprog{Lige \mw{børn} leger bedst.}
        {Подо́бный\fn{ligemand} подо́бного лю́бит.
        \mitem Рыба́к\fn{fisker} рыбака́ ви́дит издалека́
        \mitem Свой своего́ и́щет\fn{søger}.}
\ordsprog{Små \mw{børn}, små sorger, store børn, store sorger.}
        {С ма́лыми детьми́ го́ре\fn{sorg}, с больши́ми вдо́е\fn{dobbelt}.
        \mitem Без дете́й го́ре, а с детьми́ вдо́е.}
\letter{D}
\ordsprog{Hver \mw{dag} har nok i sin plage.}
        {Вся́кому дню подоба́ет\fn{være nok} забо́та\fn{omsorg} своя́.
        \mitem Довле́ет\fn{tilstrækkelig} дне́ви зло́ба\fn{onde} его́.}
\dui{dag}
\ordsprog{\mwni{Dagene} længes, vinteren strenges.}
        {\ri{Со́лнце} на \ri{ле́то}, \ri{зима́} на \ri{моро́з}.}
\ordsprog{Alt kommer for en \mw{dag}.}
        {И ши́ло\fn{syl} в мешке́\fn{pose} не ута́ишь\fn{skjule}.
%        \mitem Со́рок неде́ль хоть кого́ на
%                чи́стую во́ду вы́ведут.
        \mitem Как бы мал ого́нь\fn{ild} ни́ был, всегда́ от него́ дым\fn{røg}.}
\ordsprog{Den sidste \mw{dag} kommer for os alle.}
        {День да ночь, су́тки\fn{døgn} прочь\fn{væk}, а смерть всё бли́же\fn{nærmere}.}
\ordsprog{I \mw{dag} mig, i morgen dig.}
        {\rid{ны́не}{Ны́не}\fn{i dag/nu} меня́, \ri{за́втра} тебя́.}
\dui{daler}
\ordsprog{Hvo ikke sparer på skillingen, får aldrig \mwni{daleren}.}
        {Копе́йка рубль бережёт.
       % \mitem Кто не бережёт\fn{sparer/passer på} копе́йки, сам рубли́ не сто́ит\fn{fortjener}.
        \mitem Копе́йка к копе́йке --- проживёт\fn{nok til at leve for} и семе́йка\fn{lille familie}.
        }
\dui{datter}
\ordsprog{Vil du have \mwni{datteren}, så prøv at vinde moderen.}
        {Выбира́й\fn{vælg} коро́ву\fn{koen} по рога́м\fn{hornene}, а де́вку по рода́м\fn{slægt}.}
\dui{dovenskab}
\ordsprog{\mwni{Dovenskab} går foran, og \di{armod} følger lige efter.
        (Dovenskab er \di{roden} til alt \di{ondt}.)}
        {Ле́ность\fn{dovenskab} нахо́дит (наво́дит на) бе́дность\fn{fattigdom}.
        \mitem Лень мужика́ не ко́рмит\fn{giver mad}.}
\dui{dristig}
\ordsprog{\mwni{Dristigt} vovet er halvt vundet.}
        {Сме́лость\fn{mod} города́\fn{byer} берёт\fn{indtager}.}
\ordsprog{Den \mw{drukne} ej, som hænges skal.}
        {Кому́ сгоре́ть\fn{hænges}, тот не уто́нет\fn{drukner}.
        \mitem Кому́ суждено́\fn{dømt} опи́ться\fn{drikke (sig ihjel med gift)},
                тот о́буха\fn{ryg af knivsblad} не бои́тся.}
% slettet
%\dui{drøm}
%\ordsprog{\mwni{Drømme} er dræk, hvo der tror dem er gæk.}
%        {Куда́ ночь, туда́ и сон.}
\dui{dråbe}
\ordsprog{\mwni{Dråben} udhuler stenen.}
        {Ка́пля\fn{dråbe} и ка́мень долби́т\fn{hugge/bryde} (то́чит\fn{slider ned, skærper}.)
        \mitem Ка́пля по ка́пле ка́мень долби́т.}
\dui{due}
\ordsprog{Stegte \mwni{duer} flyver ikke lige ind i munden.}
        { Жа́реные го́луби са́ми в рот не летя́т.
        \mitem Печёные голу́бки не прилетя́т до гу́бки.}
\dui{dumhed}
\ordsprog{Mod \mwni{dumheden} kæmper selv guderne forgæves.}
        {Нет лека́рства\fn{medicin} про́тив глу́пости.
        % slettet \mitem mitem У кольца́\fn{ring} нет конца́, а у глупца́\fn{fjols} ---  нача́ла.
        }
\ordsprog{Hellere \mw{dø} end overgive sig.}
        {Скоре́е умере́ть, чем сда́ться.}
\dui{død}
\ordsprog{\mwni{Døden} skal have en årsag.}
        {Смерть причи́ну\fn{årsag} найдёт.}
\dui{død}
\ordsprog{\mwni{Døden} gør alle lige.}
        {Смерть всех поравня́ет.}
\dui{død}
\ordsprog{\mwni{Døden} står for alles dør.}
        {Смерть не разбира́ет\fn{vurderer} чи́на\fn{stand, rang},
        а ведёт равно́\fn{lige, såvel} и крестья́нина\fn{bonde},
        и дворя́нина\fn{adelsmand}.}
\dui{død}
\ordsprog{Ingen urt hjælper mod \mwni{døden}. (Mod døden vokser ingen urt).}
        {От сме́рти нет зе́лья\fn{urt}.
        \mitem От сме́рти и под ка́мнем не укро́ешься\fn{gemme sig}.}
\dui{død}
\ordsprog{\mwni{Døden} er vis, men timen er uvis.}
        {Жить наде́йся\fn{håb}, а умира́ть гото́вься\fn{forbered dig}.}
\dui{døde}
\ordsprog{Alt der fødes skal \mwni{dødes}.}
        {
%         Двух смерте́й не быва́ет, а одно́й не минова́ть\fn{undgå}.
%         \mitem
        Без сме́рти не умрёшь.}
\letter{E}
\dui{efterråd}
\ordsprog{\mwni{Efterråd} er gækkeråd.}
        {А пора́\fn{øjeblikket} ушла́, так ум\fn{forstand} пришёл.
        \mitem Была́ пора́, так не́ бы́ло ума́.}
\dui{eg}
\ordsprog{\mwni{Egen} falder også ved mange \di{hug}.
        (Mange hug fælder egen.)}
        {От отного́ уда́ра дуб не ва́лится\fn{falder}.
        \mitem За оди́н раз де́рево не ва́лится\fn{fældes}.}
\dui{eksempel}
\ordsprog{\mwni{Eksemplet} smitter.}
        {Дурно́й приме́р зарази́телен.
        \mitem Дурны́е приме́ры зарази́тельны.
        \mitem Оди́н разу́мный\fn{fornuftig} согреши́т\fn{tage fejl/synde} да мно́гих глу́пых соблазни́т\fn{frister}.}
\dui{ende}
\ordsprog{Når \mwni{enden} er god, er alting godt.}
        {Всё хорошо́, что хорошо́ конча́ется.
        %slettet \mitem Не до́брого нача́ло, а похва́лен коне́ц.
        }
% Slettet for nu
%\ordsprog{Det gode får altid en tidlig \mw{ende}.}
%        {Куса́ют и комары́ до поры́.}
\dui{enighed}
\ordsprog{\mwni{Enighed} gør stærk. }
        {В едине́нии си́ла\fn{styrke}.
        \mitem \rid{согла́сие}{Согла́сие} лу́чше ка́менных стен\fn{vægge}.
% slettet for nu        \mitem Coгла́сие к хоро́шему приво́дит.
        }
\dui{ensomt}
\ordsprog{\mwni{Ensomt} er ledsomt.}
        {И в \ri{раю́}\fn{paradis} жить то́шно\fn{kvalmende} одному́.}
\dui{erfaring}
\ordsprog{\mwni{Erfaring} er den bedste \di{læremester}.}
        {\rid{о́пыт}{О́пыт} --- лу́чший \ri{учи́тель}.}
\dui{erfaring}
\ordsprog{\mwni{Erfaring} kommer med årene.}
        {Поживёшь подо́льше\fn{længere}, узна́ешь побо́льше.}
\ordsprog{Enhver \did{yde}{yder} efter \mw{evne}, enhver nyder efter \di{indsats}.}
        {От ка́ждого по его́ спосо́бностям\fn{evne}, ка́ждому по его́ труду́\fn{arbejde}.}
\letter{F}
\ordsprog{Når man giver (rækker) \mw{fanden} en lillefinger, tager han hele hånden.}
        {Дай чёрту па́лец, он всю ру́ку отку́сит\fn{bide af} (отхва́тит\fn{rive af}).}
\ordsprog{Når man taler om \mw{fanden}, er han nærmest.}
        {О во́лке\fn{ulv} речь, а волк навстре́чь\fn{møde}.}
\dui{fanden}
\ordsprog{Man skal ikke male \mwni{fanden} på væggen.}
        {Не так стра́шен\fn{skrækkelig} чёрт\fn{fanden}, как его́ малю́ют\fn{maler}.
% slettet for nu \mitem Не шути́ чёртом: из дуби́нки убьёт(вы́палит).
}
\dui{fare}
\ordsprog{\mwni{Far} i mag.}
        {
        % slettet for nu Споко́йно, не спеши́.
        Не торопи́сь\fn{skynd dig}, а наза́д\fn{tilbage} огляди́сь\fn{se dig tilbage}.
        \mitem Что хорошо́, то не ско́ро.
        \mitem Tи́ше е́дешь, да́льше бу́дешь.}
\ordsprog{Hvad \mw{fatter} gør er altid det rigtige.}
        {Что муженёк\fn{fatter} сде́лает, то и ла́дно\fn{\ru{ла́дный}=god}.
        \mitem Нача́льник\fn{førstemanden, lederen} всегда́ прав.}
\ordsprog{Bedre \mw{fattig} med ære end rig med skam.}
        {Лу́чше бе́дность\fn{fattigdom} да че́стность\fn{renhed}, чем при́быль\fn{profit}.}
\dui{fattigdom}
\ordsprog{\mwni{Fattigdom} er ingen last.}
        {Бе́дность не поро́к.}
\dui{fattigmand}
\ordsprog{\mwni{Fattigmand} har hverken slægt eller venner.}
        {Бе́дному нигде́ ме́ста нет.}
\ordsprog{Hver mand \mw{fejer} for sin egen dør.}
        {Мети́ пе́ред свои́м крыльцо́м\fn{overdækket indgangsparti}.
        \mitem Всяк пряди́\fn{spinder} свою́ пря́жу\fn{garn}.
        \mitem В чужи́е\fn{fremmede} дела́ не су́йся\fn{stik næsen i}.
%        \mitem Знай себя́, и бу́дет с тебя́.
        \mitem Други́х не суди́\fn{døm}, --- на себя́ погляди́. }
\dui{finde}
\ordsprog{Den, som søger, \mwni{finder}.}
        {Тот и сы́щет\fn{\ru{сыска́ть}=søge}, кто и́щет\fn{\ru{сы́скать}=finde}.
%        \mitem За чем пойдёшь, то и найдёшь.
}
\dui{fisk}
\ordsprog{\mwni{Fisken} begynder først ved \mbox{hovedet} at rådne. }
        {Ры́ба начина́ет порти́ться\fn{forgå} с головы́.}
\ordsprog{De store \mw{fisk} æder de små.}
        {Больша́я ры́ба ма́ленькую целико́м\fn{fuldstændigt} глота́ет\fn{sluger}.}
\ordsprog{En tør \mw{fisker} dur ikke.}
        {Не замочи́вшись\fn{bliver våd}, ры́бки не пойма́ешь.}
\ordsprog{Enhver bærer sin \mw{fjende} i egen barm.}
        {Всяк сам себе́ враг\fn{fjende}.}
\ordsprog{En \mw{fjer} kan blive til fem høns.}
        {Де́лать из му́хи слона́\fn{elefant}.}
\dui{flid}
\ordsprog{\mwni{Flid} giver brød, dovenskab død.}
        {Без труда́ нет плода́\fn{frugt}.}
\dui{flod}
\ordsprog{Alle \mwni{floder} løber ud i havet.}
        {Все реки́ теку́т\fn{flyder} к мо́рю.
        \mitem Река́ми мо́ре стои́т.}
\ordsprog{Hellere \mw{fly} end fægte.}
        {Лу́чше оби́ду\fn{fornærmelse} терпе́ть\fn{udholde}, чем чини́ть\fn{holde rettergang} кому́.}
\ordsprog{Højt at \mw{flyve}, dybt at falde.}
        {Кто высоко́\fn{højt} лета́ет бо́льно пада́ет
        \mitem Кто высоко́ зано́сится\fn{hæver sig}, тому́ не минова́ть\fn{undgå} упа́сть.}
\dui{folk}
\ordsprog{Der er forskel på folk.}{Лю́ди (быва́ют) ра́зные.}
\dui{forbuden}
\ordsprog{\mwni{Forbuden} frugt smager bedst.}
        {Запре́тный\fn{forbuden} плод сла́док\fn{sød}.
        \mitem На запре́тный това́р\fn{vare} весь база́р.}
\dui{fordærve}
    \ordsprog{For meget af det gode \mwni{fordærver} alting.
    (For meget og for lidt fordærver alt.)}
        {Всё должно́ быть в ме́ру\fn{mål, moderation}.
        \mitem Хоро́шенького понемно́жку\fn{en lille smule}.}
\dui{forebyggelse}
\ordsprog{\mwni{Forebyggelse} er bedre end \di{helbredelse}.}
        {Береги́сь\fn{foregrib} бед\fn{ulykker}. Пока́ их нет.}
\ordsprog{Et magert \mw{forlig} er bedre end en fed proces.}
         {Худо́й мир\fn{fred} лу́чше до́брой дра́ки\fn{kamp} (ссо́ры\fn{skænderi} / бра́ни\fn{overfusning}).}
\dui{forord}
\ordsprog{\mwni{Forord} bryder ingen trætte.}
        {% slettet for nu Угово́р\fn{aftale} доро́же де́нег.
        Счёт\fn{regnskab} дру́жбы\fn{venskab} не по́ртит\fn{skader}.
        % \mitem Дава́й(те) для нача́ла внесём я́сность.
        }
\dui{forsigtighed}
\ordsprog{\mwni{Forsigtighed} er en borgmesterdyd.}
        {Опасе́ние\fn{forsigtighed} полови́на\fn{halve} спасе́нья\fn{redning}.}
\ordsprog{Hellere \mw{forspørge} sig end forgøre sig.}
        {Язы́к до Ки́ева доведёт\fn{fører til}.}
\ordsprog{Det er vanskeligt at agere \mw{forsyn} for andre.}
        {За други́х реша́ть\fn{beslutte} тру́дно.}

\ordsprog{Man har ikke \mw{fred} længere end ens nabo vil.}
        {Сосе́д не захо́чет, так и ми́ру не бу́дет.
        \mitem Хоро́ший сосе́д --- клад\fn{skat}.
        \mitem С сосе́дом дружи́\fn{gør dig ven}, а забо́р\fn{plankeværk} чини́\fn{vedligehold}.}
\ordsprog{Bedre uret \mw{fred} end en retfærdig krig.}
        {Худо́й мир лу́чше до́брой дра́ки (ссо́ры)(бра́ни).}
\dui{frihed}
\ordsprog{\mwni{Frihed} er det bedste guld.}
        {Свобо́да доро́же зо́лота.}

\dui{frygt}
\ordsprog{\mwni{Frygten} har tusind øjne.}
        {У стра́ха глаза́ велики́\fn{store}.}
\dui{frækhed}
\ordsprog{\mwni{Frækhed} belønnes.}{Сме́лость\fn{dristighed} города́ берёт\fn{indtager}.}
\ordsprog{Frænde er \mw{frænde} værst.}
        {Свой своему́ ху́дший враг.}
\ordsprog{En \mw{fugl} i hånden er bedre end ti på taget.
  (Hellere en \di{fisk} i hånden end to ved stranden).}
        {Лу́чше сини́ца\fn{blåmejse} в рука́х, чем жура́вль\fn{trane} в не́бе.
%     slettet for nu   \mitem Не сули́ журавля́ в не́бе, а дай сини́цу.
        % slettet for nu Лу́чше кро́шечное\fn{lillebitte} сбереже́ние\fn{spareskilling}, чем больша́я при́быль\fn{profit}.
        \mitem Лу́чшe оди́н воробе́й\fn{gråspurv} в карма́не, чем деся́ток на кры́ше\fn{taget}.
        \mitem Сини́ца в рука́х лу́чше соловья́\fn{nattergal} в лесу́.

}
\dui{fugl}
\ordsprog{Man kender \mwni{fuglen} på fjerene.}
        {Ви́дно пти́цу по полёту\fn{flugten}.}
\ordsprog{Hver \mw{fugl} synger med sit næb.}
        {У вся́кой пти́зы своя́ пе́сня.
        \mitem Вся́кая пти́ца свои́ пе́сни поёт.
        \mitem Всяк поёт как уме́ет.
        % slettet for nu \mitem Ка́ждый поступа́ет\fn{optræder} по со́бственному\fn{sin egen} усмотре́нию\fn{vurdering}.
        \mitem Всяк свои́м го́лосом поёт.
        \mitem Всяк свои́м умо́м живёт.}
\ordsprog{Det er en dårlig \mw{fugl}, der besudler sin egen rede.}
        {Пло́ха та пти́ца, кото́рая га́дит в своём гнезде́.}
\ordsprog{Enhver familie har sit sorte \mw{får}.}
        {Нет семьи́ без уро́да\fn{vanskabning}.
% slettet        \mitem Сеья́ не без уро́да
}
\letter{G}
\ordsprog{Man skal ikke \mw{gabe} højere end man kan bide.}
        {Вы́бирай епанчу́\fn{hvid kappe uden ærmer} по плечу́\fn{skulderen}.}
\ordsprog{Intet er så \mw{galt}, at det ikke er godt for noget.}
        {Нет ху́да\fn{onde} без добра́\fn{gode}.
% slettet        \mitem Не бы́ло бы сча́стья, да несча́стье помогло́\fn{hjalp}.
}
%%% \ordsprog{Som de \mw{gamle} sjunge, så kvidre de unge.}{}
\ordsprog{De \mw{gamle} til \di{råd}, de \di{unge} til \di{dåd}.}
        {Молодо́й на би́тву\fn{slag}, а ста́рый на ду́му.}
\ordsprog{Ingen er for \mw{gammel} til at lære.}
        {Учи́ться\fn{studere} никогда́ не по́здно.}
\ordsprog{Én \mw{gang} er ingen gang.}
        {Оди́н раз не в счёт\fn{optælling}.}
\dui{gemme}
\ordsprog{Den, som \mwni{gemmer} til natten, gemmer til katten.}
        {Не сто́ит оставля́ть\fn{гемме} ничего́ на пото́м.
%%Нет ничего́ та́йного, что не ста́ло бы я́вным.
}
\dui{gemt}
\ordsprog{\mwni{Gemt} er ikke glemt.}
        {Закры́то, но не забы́то.}
\ordsprog{Gjort \mw{gerning} står ikke til at ændre.}
        {Сде́ланного\fn{gjorte} не воро́тишь\fn{gøres om}.}
\dui{gerrig}
\ordsprog{Den \mwni{gerrige} er altid fattig.}
        {Скупо́й\fn{gerrige} бога́ч\fn{rigmand} бедне́е\fn{fattigere}
        ни́щего\fn{tigger}.}
\dui{gerrighed}
% \ordsprog{\mwni{Gerrighed} og øjet mættes aldrig.}
%         {У скупо́го, что бо́льше
%          де́нег, то бо́льше го́ря\fn{sorg}.\fn{Samme betydning?}}
\ordsprog{Skal der være \mw{gilde}, så lad der være gilde.}
        {Кути́ть\fn{svirre} - так кути́ть.
       % slettet for nu. \mitem Де́лать - так де́лать как сле́дует\fn{hør og bør}.
        \mitem Раз пошла́ така́я пья́нка\fn{brandert}.
        \mitem Режь\fn{skær} после́дний\fn{sidste} огуре́ц\fn{agurk}.
       }
\dui{glæde}
\ordsprog{Når \mwni{glæden} er i stuen, er sorgen i forstuen.}
         {За ра́достью го́ре\fn{sorg} по пята́м\fn{i hælene} хо́дит.}
\ordsprog{En liden \mw{gnist} kan antænde en hel skov.}
        {Ма́ла и́скра\fn{gnist} города́ пожига́ет\fn{antænder}.
        \mitem От де́нежной све́чки\fn{kjærte} Москва́ сгоре́ла\fn{brændte}.}
\ordsprog{Af liden \mw{gnist} vorder ofte stor ild.}
        {От ма́лой и́скры, да большо́й пожа́р.
%        \mitem От ма́лой и́скры сыр-бо́р загоре́лся
%                (загора́ется).
                }
\ordsprog{Den, der graver en \mw{grav} for andre, falder selv i den.}
        {Не рой\fn{grav} друго́му я́му\fn{hul/grav}, сам в неё попадёшь\fn{falder i}.
        \mitem Кто друго́му я́му ро́ет, сам в неё попадёт.}
\ordsprog{Man skal ikke save den \mw{gren} over, som man sidder på.}
        {Не руби́\fn{skær} сук\fn{gren}, на кото́ром сиди́шь.}
\dui{smed}
\ordsprog{Hvad der kurerer en \mw{grovsmed}, \mbox{slår}
          en \di{skrædder} ihjel.}
        {Что поле́зно\fn{nyttigt} одному́, то вре́дно\fn{skadeligt} друго́му.
        \mitem Что ру́сскому здоро́во, то не́мцу смерть.
        \mitem Что одному́ здоро́во, друго́му смерть.}
\dui{græs}
\ordsprog{Mens \mwni{græsset} gror, dør horsemor.}
        {Пока́ трава́ вы́растет, кобы́ла\fn{hoppe} \mbox{сдо́хнет}.
%        \mitem Пока́ со́лнце взойдёт, роса́ о́чи вы́ест.
}
\dui{græsset}
\ordsprog{\mwni{Græsset} er altid grønnere på den anden side af hækken (Naboens \mw{høne} lægger altid større æg).}
         {
         В чужо́м дворе́\fn{\ru{двор} = gårdsplads} ку́рица гу́сем\fn{\ru{гусь} = gås} ка́жется.
         \mitem Зави́стливый\fn{misundelig} по чужо́му\fn{andens}
         сча́стью\fn{lykke} со́хнет\fn{visner}.}
\dui{gråspurv}
\ordsprog{Man skal ikke skyde \mwni{gråspurve} med kanoner.}
        {Не стреля́й\fn{skyd} из пу́шки\fn{kanon} по воробья́м\fn{spurve}.}
\dui{gud}
\ordsprog{Hjælp dig selv, så hjælper dig \mwni{Gud}.}
        {Бе́режного\fn{forsigtig} Бог бережёт\fn{beskytter}.
        \mitem Бог-то бог, да не будь сам плох.
%        \mitem На Бо́га наде́йся\fn{sæt lid til},
%                а сам не проша́й (лени́сь).
                }
\ordsprog{Det er ikke alt \mw{guld}, der glimrer.}
        {Не всё то \ri{зо́лото}, что блести́т.}
\dui{guld}
\ordsprog{\mwni{Guld} skinner, omend det ligger i skarn.}
        {Зо́лото и в грязи́ блести́т.}
\dui{guldnøgle}
\ordsprog{\mwni{Guldnøgle} lukker alle døre op.}
        {Зо́лото не бог, а ми́лует\fn{frelser}.}
\dui{gæld}
\ordsprog{\mwni{Gæld} er for at betales.}
        {Долг платежо́м кра́сен\fn{smuk}.}
\ordsprog{Ubuden \mw{gæst} hører ej til fest.}
         {% Незва́ные \ri{го́сти} гло́жут'\fn{afgnaver} и ко́сти\fn{knogler}.
        Незва́ный гость ху́же\fn{værre} тата́рина\fn{tatar}.}
\dui{gæst}
\ordsprog{Når \mwni{gæsten} er kærest, skal han takke af.}
        {Пора́ гостя́м и честь\fn{ære, høflighed} знать.}
\ordsprog{En \mw{gæst} og en fisk lugter ilde den tredje dag.}
        {Гость до трёх дней.}
\ordsprog{Det er en kær \mw{gæst}, som sjældent kommer.}
        {
        Ре́дкое свида́ние\fn{møde}, прия́тный гость.
        \mitem Хоро́ш гость, ко́ли\fn{hvis, dersom} ре́дко хо́дит.
        \mitem Мил гость, что недо́лго гости́т.
        }
\dui{gø}
\ordsprog{Den hund, der \mwni{gøer}, bider ikke.}
        {Соба́ка, кото́рая ла́ет, не куса́ет.
        \mitem Не бо́йся соба́ки, что
        ла́ет (а бо́йся той, что молчи́т да
        хвосто́м\fn{halen} виля́ет\fn{logrer}).}
\ordsprog{Hvad du ikke vil, man skal \mw{gøre} mod dig,
        det skal du ikke gøre mod andre.}
        {Чего́ в друго́м не лю́бишь, того́ и сам не
        де́лай.}
\letter{H}
\dui{handle}
\ordsprog{Hvor der \mwni{handles}, der spildes.}
        {Лес ру́бят\fn{fælde}, ще́пки\fn{spåner} летя́т.
        %\mitem Незамочи́вшись, ры́бки не пойма́ешь.
        %\mitem Не чёрт, не мят --- не бу́дет кала́ч\fn{hvedebrød}.
        }
\dui{hastværk}
\ordsprog{\mwni{Hastværk} er \di{lastværk}.}
        {Ти́ше\fn{roligere} е́дешь, да́льше бу́дешь.}
\dui{held}
\ordsprog{\mwni{Heldet} følger de tossede.}
        {Дура́к спит\fn{sover}, а сча́стье в голова́х сиди́т (стои́т).}
\dui{held}
\ordsprog{\mwni{Held} i spil, uheld i kærlighed.}
        {Счастли́в игро́й, да несчастли́в жено́й.
        \mitem Кому́ везёт в игре́ (в ка́ртах), тому́ не везёт в любви́.}
\ordsprog{Vejen til \mw{helvede} er brolagt med gode forsætter.}
        {Доро́га в ад\fn{helvede} вы́мощена\fn{brolagt}
        благи́ми\fn{velmenende} наме́ренями\fn{forsætter}.}
\dui{hensigt}
\ordsprog{\mwni{Hensigten} helliger midlet.}
        {Цель\fn{målet} опра́вдывает\fn{retfærdiggør} сре́дства\fn{midler}.}
\ordsprog{Det er svært at være \mw{her} og der og alle vegne.}
        {Всю́ду\fn{alle steder hen} не поспе́ешь\fn{når}.}
\dui{herre}
\ordsprog{Som \mwni{herren} er, så følge ham hans svende.}
        {
        Како́в царь, тако́в наро́д.
        \mitem Како́в поп\fn{præst}, тако́в и прихо́д\fn{sogn}.
        \mitem Како́в ба́тюшка\fn{fader}, тако́в и сын.
        \mitem Оте́ц рыба́к\fn{fisk} - и де́ти в во́ду
        смо́трят\fn{kigger}.
        \mitem Како́в хан\fn{kahn}, такова́ орда́\fn{horde}
        }
\dui{herre}
\ordsprog{Ingen kan tjene to \mwni{herrer}.}
        {На двух госпо́д служи́ть\fn{tjene}, ни одному́ не угоди́ть\fn{gøre tilpas}.}
\ordsprog{\mw{Herren} gav, Herren tog.}
        {Бог дал, Бог и взял.}
\ordsprog{Fra en høj \mw{hest} falder man dybt.}
        {
        Высоко́ по́днял, да ни́зко опусти́л.
        \mitem С высо́кой ло́шади больне́е\fn{desto mere smertefuldt} па́дать.
        }
 % \dui{hest}
 %\ordsprog{Enten skal man køre \mwni{hest\-e\-ne} eller give tøjlerne fra
 %        sig.}
 %        {Взя́лся за гуж, не говори́, что не дюж.}
\ordsprog{Man skal ikke skue given \mw{hest} i munde.}
        {Дарёному(даро́вому) коню́ в зу́бы не смо́трят.}
\ordsprog{En \mw{hest} kan falde på fire ben, (hvorfor så ikke et
        menneske på to?)}
        {Конь о четырёх нога́х, да и тот спотыка́ется\fn{snuble}.}
% slettet \ordsprog{Mands vilje, mands \mw{himmerige}.}
%        {Своя́ во́ля, ли́бо\fn{enten ... eller} рай, ли́бо ад.
%        \mitem Охо́та\fn{?} пу́ще (стра́шней) нево́ли.
%        }
\ordsprog{Man kan tvinge hesten til brønden, men man kan ikke tvinge den til at drikke.}{Не гоня́й\fn{\ru{гоня́ть}=drive} ло́шадь к воде́, е́сли ей пить не хо́чется.}
\ordsprog{Øst, vest, \mw{hjemme} bedst. (Hvem der vil ligge godt og sidde godt, må blive \mw{hjemme}).}
        {В гостя́х хорошо́, а до́ма лу́чше.
        \mitem Своя́ ха́тка\fn{hytte} --- родна́я ма́тка\fn{skød, kære mor}.
        \mitem Своя́ хи́жина\fn{hytte} лу́чше чужи́х\fn{fremmedes} пала́т\fn{palads}.
        \mitem Дон, Дон, а лу́чше дом.}
\dui{hjerte}
\ordsprog{Hvad \mwni{hjertet} er fuldt af, løber munden over med.
(Af hjertets overflødighed taler munden.)}
        {От избы́тка\fn{overskud} се́рдца, уста́\fn{munden} глаго́лют\fn{taler}.
        \mitem У кого́ что боли́т\fn{lider}, тот о том и говори́т.}
\ordsprog{Hvor der er \mw{hjerterum}, er der også husrum.}
        {В тесноте́\fn{tranghed, smalhals}, да не в оби́де\fn{fornærmelse}.
        \mitem Те́сно\fn{snævert}, но ую́тно\fn{hyggeligt}.
        }
\ordsprog{Det sletteste \mw{hjul} på vognen skriger mest.}
        {Худо́е колесо́\fn{hjul} пу́ще\fn{mere} скрипи́т\fn{knirker}.}
\ordsprog{Hurtig \mw{hjælp} er dobbelt hjælp.}
        {Даю́ший во́время, даёт вдво́е.
        \mitem Кто ско́ро помо́г, тот два́жды помо́г.}
\dui{honning}
\ordsprog{\mwni{Honning} i munden, galde i hjer\-tet.}
        {На языке́ мёд, а в се́рдце лёд\fn{is}.}
%\ordsprog{Hvor \mw{honning} er, der samles (el. sankes) \di{fluer}.}
%        {Где мёд, там и му́хи.
%        \mitem Будь лишь мёд, мух мно́го нальнёт\fn{samles?}.}
\dui{hoved}
\ordsprog{Så mange \mwni{hoveder}, så mange \di{sind}.}
        {Ско́лько голо́в, сто́лько умо́в.}
\dui{hoved}
\ordsprog{Hvad man ikke har i \mwni{hovedet}, må man have i benene.}
        {Дурна́я голова́ нога́м поко́я\fn{ro} не даёт.
        \mitem За глу́пой голово́й и нога́м не поко́й.}
\dui{hovmod}
\ordsprog{\mwni{Hovmod} står for fald.}
        {Высоко́ лета́ешь, да ни́зко сади́шься\fn{sætte sig}.
        \mitem Не надува́йся\fn{blæse sig op}, ло́пнешь\fn{revner}.
        \mitem Кто высоко́ зано́сится\fn{flyver}, тому́ не минова́ть\fn{undgå}
                упа́сть.
        \mitem Кто высоко́ лета́ет, тот ни́зко па́дает.
        \mitem Горды́ня\fn{hovmod} до добра́ не доведёт\fn{fører.}}
\dui{huld}
\ordsprog{\mwni{Huld} er mere værd end \di{guld}.}
        {Здоро́вье --- то же зо́лото.}
\dui{hund}
 %\ordsprog{Mange \mwni{hunde} er \did{hare}{harens} død.}
 %       {Си́ла\fn{styrke} соло́му\fn{halm} ло́мит\fn{?}.}
 \dui{hund}
 \ordsprog{Man kan ikke skue \mwni{hunden} på hårene.}
         {Вне́шность\fn{udseende} --- обма́нчива\fn{vildledende}.}
\ordsprog{Død \mw{hund} bider ikke.}
        {Мёртвая соба́ка не куса́ет.}
\dui{hund}
\ordsprog{Man skal ikke lære gamle \mwni{hunde} nye kunster.}
        {Не учи́ ры́бу пла́вать.
        \mitem Не учи́ учёного\fn{lærte}.}
\dui{hund}
\ordsprog{Tiende \mwni{hunde} bider værst.}
        {Молчá --- соба́ка исподтишка́\fn{i det skjulte} хвата́ет\fn{snapper}.}
\dui{hund}
\ordsprog{Hvo, der går i seng med \mwni{hunde}, han står op med
        lopper.}
        {С соба́кой ля́жешь, с блоха́ми\fn{lopper} ста́нешь.
        \mitem С кем поведёшься\fn{\ru{повести́сь}=blive ven med}, от того́ и наберёться\fn{\ru{набра́ться}=samle op}.}
\dui{hus}
\ordsprog{Andre \mwni{huse}, andre sæder.}
        {Что го́род, то но́ров\fn{skik}, что дере́вня, то обы́чай\fn{vane}.}
%        \mitem Нет таки́х трав, что́бы знать чужо́й\fn{fremmed} нрав\fn{gemyt sæder/skik}.
\ordsprog{Lykkeligst at \mw{hvile} på er fuldendte gerning.}
        {По́сле трудо́в сла́док поко́й\fn{hvile}.}
\dui{hæler}
\ordsprog{\mwni{Hæleren} er lige så god som stjæleren.}
        {
        % Во́ру\fn{tyv} потака́ть\fn{ser gennem fingre med, dækker over}, что самому́ ворова́ть.
        Ута́йщик\fn{hæler} - тот же вор.
        \mitem Вор по во́ре след кро́ет\fn{\ru{крыть}=dække}.
        }
\dui{høflighed}
\ordsprog{\mwni{Høflighed} koster ingen penge.}{
Ве́жливость\fn{høflighed} нам ничего́ не сто́ит (но це́нится\fn{værdsat} до́рого).
% slettet for nu \mitem Честь безче́стья лу́чше.
}
 %\ordsprog{Høg over \mw{høg}.}
 %        {Нашла́ коса́ на ка́мень.
 %        \mitem Ты концо́м, а он кольцо́м.}
\ordsprog{Blind \mw{høne} finder også et korn.}
        {И слепа́я ку́рица мо́жет найти́
                зёрнышко (зерно́).}
\ordsprog{Kloge \mw{høns} gør også i nælderne.}
        {На вся́кого мудреца́\fn{klog mand} дово́льно\fn{ret så meget}
        простоты́\fn{enfoldighed}.
        \mitem И на стару́ху быва́ет прору́ха.
        \mitem И на мо́лодца\fn{ung mand} быва́ет опло́х\fn{fejl} (опло́шность).
}
\ordsprog{Den, der ikke vil \mw{høre}, må føle.}
        {Не послу́шаешься, попла́тишься\fn{\ru{плати́ться}=betale}.
        \mitem Кого́ слова́ не беру́т\fn{tager},
        с того́ шку́ру\fn{pelsen} деру́т\fn{\ru{драть}=flå}.}
\dui{håb}
\ordsprog{\mwni{Håbet} er lysegrønt.}
        {Наде́жда\fn{håb} умира́ет\fn{dør} после́дней.
        \mitem Сча́стье\fn{lykke} ско́ро\fn{snart} покида́ет\fn{forlader}, а до́брая
                наде́жда --- никогда́.}
\ordsprog{Den ene \mw{hånd} \di{vasker} den anden.}
        {Рука́ ру́ку мо́ет.
                (И о́бе бе́лы живу́т)
 %        \mitem Плут плута́ кро́ет.
}
\ordsprog{Mange \mw{håndværk} fordærver \di{mesteren}.}
        {
        % slettet for nu От ску́ки ма́стер на все ру́ки.
        Кто за всё хвата́ется\fn{tager fat på, griber}, тот ничего́ не сде́лает.}
\dui{hård}
\ordsprog{\mwni{Hårdt} mod hårdt.}
        {Нашла́ коса́\fn{le} на ка́мень\fn{sten}.
        \mitem Ты концо́м\fn{ende?}, а он кольцо́м\fn{ring}.}
\letter{I}
\ordsprog{Når der går \mw{ild} i gamle huse, brænder de ned til
grunden.}
        {Седина́\fn{grå hår} в бо́роду\fn{skægget}, а бес\fn{fanden} в ребро́\fn{ribben}.}
\dui{ild}
\ordsprog{Man skal ikke lege med \mwni{ilden}.}
        {Не игра́й с огнём.}
\letter{J}
\dui{jage}
\ordsprog{Den, som \mwni{jager} to \di{harer}, får ingen.}
        {За двумя́ за́йцами\fn{harer} пого́нишься\fn{jager}, ни одного́ не пойма́ешь.
        \mitem Двух за́йцев гоня́ть\fn{jage}, ни одного́ не пойма́ть}
\dui{jern}
\ordsprog{Man skal smede, mens \mwni{jernet} er varmt.}
        {Куй\fn{smed} желе́зо, пока́ горячо́\fn{hedt}.}
\ordsprog{Af \mw{jord} er du kommet, til jord skal du blive.}
        {Из земли́ вы́шел и в зе́млю оты́деши\fn{?}.}
\ordsprog{God \mw{jæger} får altid bytte.}
        {На ловца́\fn{fanger} и зверь\fn{vildt dyr} бежи́т\fn{løber}.}
\letter{K}
\ordsprog{Man skal ikke skære alle over en \mw{kam}.}
        {Нельзя́ стричь\fn{skære} всех под одну́
        гребёнку\fn{kam}.
        \mitem Нельзя́ ста́вить\fn{stille} всех на одну́ до́ску\fn{bræt}.
        % slettet for nu \mitem Не сле́дует вали́ть\fn{kaste} всё в одну́ ку́чу\fn{bunke}.
        %\mitem В одну́ пётельку всех пуго́вок
        %       (пу́говиц) не у\-стег\-нёшь.
        %\mitem Не всё терь на свой арши́н.
        }
\ordsprog{Der er brådne \mw{kar} i alle lande.}
        {В семье́ не без уро́да\fn{vandskabning}.
               }
\dui{kat}
\ordsprog{Om \di{natten} (i \di{mørke}) er alle \mwni{katte} grå.}
        {Но́чью все ко́шки се́ры (чёрны).
        \mitem Но́чью и уро́д\fn{vandskabning} краса́вец.}
\dui{kat}
\ordsprog{Gale \mwni{katte} får revet skind.}
        {На задо́рном\fn{kåd} буя́не\fn{ballademager} век\fn{evighed} шку́ра\fn{huden} в изъя́не\fn{mangel/defekt}.}
\dui{kat}
\ordsprog{Når \mwni{katten} er ude, spiller musene på bordet.}
        {Когда́ нет кота́ в дому́, игра́ют мы́шки по столу́.
        \mitem Кота́ до́ма нет, мыша́м раздо́лье\fn{leger}].
    % slettet for nu   \mitem Ко́шки грызу́тся\fn{kæmper indbyrdes}, мыша́м раздо́лье.
        }
\ordsprog{Sovende \mw{kat} fanger ingen mus.}
        {Еда́ не достаётся\fn{opnås} лёжа\fn{liggende}.
        \mitem Лёжа хле́ба/пи́щи\fn{mad} не добу́дешь.}
 %\ordsprog{En \mw{kat} må se på en konge.}
 %        {И ко́шка мо́жет смотре́ть на короля́.
 %        \mitem Ка́здый челове́к до́лжен облада́ть чу́вством
 %                со́бственного досто́инства.}
\dui{kejser}
\ordsprog{Hvor intet er, har \mwni{kejseren} tabt sin ret.}
        {На нет и суда́\fn{domstol} нет}
\dui{kende}
\ordsprog{\mwni{Kend} dig selv!}
        {Познава́й себя́!}
\ordsprog{Man skal ikke spise \mw{kirsebær} med de store (med herrebørn).}
        {Не ешь с больши́ми ви́шен.
 %        \mitem Не в свои́ са́ни не сади́сь.
 %        \mitem С боя́рами не ешь ви́шен (костьми́ закида́ют).
 }
\dui{klog}
\ordsprog{Den \mwni{kloge} mand tisser ikke mod vinden.}
        {Про́тив ве́тра не наду́ешься\fn{blæse}.}
\dui{klog}
\ordsprog{Den \mwni{kloge} giver efter.}
        {У́мный уступа́ет\fn{giver efter}.
        \mitem Смирне́е\fn{fredeligste} --- при́быльнее\fn{får mest ud af det}.}
\dui{klokke}
\ordsprog{Små \mwni{klokker} har også lyd.}
        {Невели́к\fn{lille} сверчо́к\fn{fårekylling}, да гро́мно поёт.}
\dui{klæder}
\ordsprog{\mwni{Klæder} skaber folk.}
        {Оде́жда кра́сит челове́ка.
        %\mitem Тот и умён, кто бога́то наряжён.
        }
\dui{klæder}
\ordsprog{\mwni{Klæder} skaber ikke folk.}
        {Не всяк мона́х\fn{munk}, на ком клобу́к\fn{munkekåbe}.
        \mitem Свинья́ и в золото́м оше́йнике\fn{halskæde} --- свинья́.}
%%\dui{knogle}
%%\ordsprog{For dem der kommer for sent er der kun \mwni{knoglerne}.}{}
 \dui{ko}
 \ordsprog{Der er flere røde \mwni{køer} end \di{præstens}.}
         {
         Свет не кли́ном\fn{kile} сошёлся\fn{\ru{сойти́сь}=komme sammen, mødes} (на э́том).
           %Э́то к вам (к тебе́) не отно́сится.
           %        \mitem Ты тут не при чём.
         %(Э́того) хоть пруд пру́ди.
         }
\ordsprog{Sort \mw{ko} giver hvid mælk.}
        {Чёрная коро́ва, да бе́лое молоко́.}
\dui{kok}
\ordsprog{For mange \mwni{kokke} fordærver maden.}
        {Де́сять поваро́в то́лько щи пересливaют\fn{oversalter}.
        \mitem У семи́\fn{syv} ня́нeк\fn{barnepiger}, дитя́\fn{børn} без гла́зу.}
\dui{komme}
\ordsprog{Hvad der \mwni{kommer} let går let.}
        {Что легко́ на́жито\fn{tjene/samle},
        то легко́ и про́жито\fn{opbrugt, forbrugt}.}
\ordsprog{Kan du vente, kan du blive \mw{konge} af Sverige.}
        {Терпе́ние и труд всё перетру́т\fn{\ru{тереть} "overkomme", egl. gnide hul på, gennempudse}.}
\ordsprog{Mangen har gode \mw{kort} på hånden, vidste han blot at
                spille dem.}
        {У мно́гих был тала́нт, е́сли бы
                зна́ли, как им по́льзоваться.}
\dui{kost}
\ordsprog{Nye \mwni{koste} fejer bedst.}
        {Но́вая метла́\fn{kost} чи́сто\fn{rent}/(хорошо́) метёт\fn{fejer}.}
\dui{krage}
\ordsprog{\mwni{Krage} søger mage.}
        {
        % slttet for nu Орёл\fn{ørn} орла́ плоди́т\fn{avler}, а сова́\fn{ugle} сову́ роди́т\fn{føder}.
        Рыба́к\fn{fisker} ви́дит издалека́\fn{fra lang afstand} рыбака́.}
\dui{krukke}
\ordsprog{\mwni{Krukken} går så længe til vands, at den kommer
        hankeløs hjem.}
        {Пова́дился\fn{lagde sig vane til} кувши́и\fn{krukke} по́ воду ходи́ть\fn{gå på vandet}, там ему́ и
        го́лову сложи́ть\fn{\ru{сложи́ть го́лову}=give sit liv, falde}.}
\dui{krukke}
\ordsprog{Små \mwni{krukker} har også ører.}
        {И у стен\fn{væg} есть у́ши.}
\ordsprog{Når \mw{krybben} er tom, bides hest\-e\-ne.}
        {У пусто́го коры́та\fn{krybbe, egtl. kar} и ко́ни\fn{hest} грызу́тся\fn{bides}.}
\ordsprog{Man må \mw{krybe} før man kan gå.}
        {Всё даётся\fn{gives} не сра́зу.
       % \mitem Мастерства́ за плеча́ми не но́сят.
       }
%  %\ordsprog{Man kan lige så godt springe som \mw{krybe} i det.}
%  %      {Не мытьём, так ка́таньем.}
\dui{kundskab}
\ordsprog{\mwni{Kundskab} er \di{magt}}
        {Зна́ние --- си́ла.}
\dui{kunne}
\ordsprog{Man \mwni{kan} hvad man \di{vil}.}
        {Что хочу́, то и могу́.
        \mitem Где хоте́ние\fn{vilje}, там и уме́нье\fn{forstand/evne}.}
\ordsprog{Ingen list som \mw{kvindelist} / Altid har kvinder svig
        under skød.}
        {Нет в лесу́ сто́лько повёрток\fn{drejning, sving} ско́лько у ба́бы увёрток\fn{udflugter}.}
\dui{kvinder}
\ordsprog{\mwni{Kvinder} har langt hår og kort forstand (Kvinder har korte sind under lange klæder).}
        {У ба́бы во́лос до́лог, да ум\fn{forstand} ко́роток.}
\ordsprog{Gammel \mw{kærlighed} ruster ikke.}
        {Ста́рая любо́вь не ржаве́ет.
        \mitem Ста́рая любо́вь до́льго по́мнится\fn{huskes}.}
\ordsprog{I krig og kærlighed gælder alle kneb.}{На войне́\fn{krig} и в любви́ все сре́дства\fn{midler} хороши́.}
\dui{kærlighed}
\ordsprog{\mwni{Kærlighed} gør blind.}
        {Любо́вь --- слепа́\fn{blind} %(поведёт до беды́ и попа́).
        }
\dui{kærlighed}
\ordsprog{\mwni{Kærlighed} overvinder alt.}
        {Любо́вь всё побежда́ет\fn{vinder}.
        \mitem Любо́вь за де́ньги не ку́пишь.}
\letter{L}
\dui{lad}
\ordsprog{\mwni{Lad} er den der \di{magsvejr} laster.}
         {Лени́вец\fn{den dovne} и хоро́шую пого́ду\fn{vejr}
         брани́т\fn{forbander}.}
\ordsprog{Livet er ikke lutter \mw{lagkage}.}
         {
        Жизнь --- не слошно́й\fn{lutter} пра́здник\fn{festdag/helligdag}.
        \mitem Не всё коту́ ма́сленица\fn{det søde liv}, придёт и вели́кий пост\fn{faste}.
%        \mitem Жизнь --- прожи́ть --- не по́ле перейти́.
%        \mitem Жизнь нелёгкая.
}
\dui{lastværk}\dui{hastværk}
\ordsprog{Hastværk er \mwni{lastværk}.}
        {Ти́ше е́дешь да́льше бу́дешь.
        \mitem Поспеши́шь\fn{skynd dig} --- люде́й насмеши́шь\fn{grine ad dig}.}
\ordsprog{For megen \mw{latter} ender med gråd.}
              {Кто в суббо́ту смеётся, в воскре́сенье пла́кать бу́дет.}
\dui{laurbær}
\ordsprog{Man bør ikke hvile på sine \mwni{laurbær}.}
              {Не сле́дует почива́ть\fn{hvile} на ла́врах.}
\dui{lediggang}
\ordsprog{\mwni{Lediggang} er roden til alt ondt.}
        {Лень\fn{dovenskab} --- мать всех поро́ков\fn{laster}.
        \mitem Пра́здность\fn{lediggang} - мать поро́ков.
        \mitem На безде́льи\fn{lediggang} дурь\fn{dumhed} в го́лову ле́зет\fn{kryber}.}
\dui{leg}
\ordsprog{Man skal holde op mens \mwni{legen} er god.}
        {Шути́\fn{spøg} до тех пор\fn{indtil nu}, пока́ кра́ска\fn{farve} в лицо́ не войдёт.
%        \mitem Останови́ться\fn{stands} во́время.
        \mitem Ешь, не доеда́й, пей, не допива́й.}
\dui{lejlighed}
\ordsprog{\mwni{Lejlighed} gør tyve.}
         {Пло́хо не клади́\fn{\ru{класть}=anbringe (tingene) dårligt}, во́ра\fn{tyv} в грех\fn{synd} не вводи́\fn{før}.}
% slettet \ordsprog{Grib \mw{lejlighed}
%               fortil, hvor den har hår,
%               og ej bagtil, hvor den er skaldet.}
%               {Со́лнышко\fn{solen} нас не дожида́ется\fn{venter}.
%        %\mitem Клюёт, так не зева́й.
%        }
\ordsprog{Med \mw{lempe} når man længst.}
         {Мёдом\fn{honning} бо́льше мух\fn{fluer} нало́вишь\fn{fanger}.}
\ordsprog{Den \mw{ler} bedst, som ler sidst. }
        {Хорошо́ смеётся тот, кто смеётся после́дним.}
\dui{leve}
\ordsprog{Den der \mwni{lever}, får at se.}{Поживём --- уви́дим.}
\dui{leve}
\ordsprog{Man skal spise for at \mwni{leve}, ikke leve for at spise}
{мы еди́м, что́бы жить, а не живём, что́бы есть.}
\ordsprog{Hellere \mw{revne} end \di{levne}.}
        {Лу́чше перее́сть\fn{overspise}, чем недое́сть\fn{ikke spise op}.}
\dui{lidt}
\ordsprog{\mwni{Lidt} er bedre end intet.}
        {Лу́чше немно́го, чем ничего́.
        \mitem Ма́лый бары́ш\fn{profit} лу́чше большо́го накла́да\fn{tab}.
        \mitem Лу́чше ма́ленькая ры́бка,
        чем большо́й тарака́н\fn{kakerlak}.}
\dui{lidt}
\ordsprog{\mwni{Lidt} har også ret.}
         {
         % slettet Хоть\fn{på trods af} грош\fn{2 kopek}, да свой.
         Лу́чше немно́го, чем ничего́
         }
\ordsprog{Lige for \mw{lige}, hvis \di{venskab} skal holdes.}
         {Услу́га\fn{tjeneste} за услу́гу.
         \mitem О́ко за о́ко, зуб за зуб.
        \mitem Ча́ще\fn{oftere} счёт\fn{regnskab}, до́льше\fn{længere} (крепче) дру́жба\fn{venskab}.}
\ordsprog{Som man reder, så \mw{ligger} man.}
        {Как посте́лешь, так и поспи́шь\fn{sover}.
        \mitem Что посе́ешь\fn{sår}, то и пожнёшь\fn{høste}.
        \mitem Какова́ посте́ль\fn{seng}, тако́в и сон\fn{søvn}.}
\ordsprog{Man skal helst lade sit skidne \mw{linned} vaske hjemme.}
              {Сор из избы́ не выноси́ (вымета́й).}
\dui{livet}
\ordsprog{\mwni{Livet} er kort, \di{kunsten} er lang.}
{Жизнь коротка́, \ri{иску́сство} ве́чно.\fn{evig}.}
\ordsprog{Lov er \mw{lov} og lov skal holdes.}
              {
              % Зако́н\fn{lov} сле́дует\fn{bør} соблюда́ть\fn{følge}.
              Зако́н\fn{lov} есть зако́н.
              }
\dui{lov}
\ordsprog{\mwni{Loven} er \di{ærlig}, holden
             besværlig / En ting er at \di{love}, en anden at \di{holde}.}
        {Одно́ де́ло обеща́ть\fn{love}, друго́е --- выполня́ть\fn{holde}.
\mitem Обеща́ть и сло́во сдержа́ть\fn{holde}, как не́бо\fn{himmel} и земля́\fn{jord}.
        % Slettet
  %%        \mitem Не да́вши сло́ва --- крепи́сь\fn{fæstne} --- а
  %%               да́вши держи́сь.
        \mitem От сло́ва до де́ла --- сто перего́нов\fn{\ru{перего́н}=jernbanestrækning (et godt stykke vej)}.
        \mitem Обеща́нного три го́да ждут.
        \mitem Мно́го сули́т\fn{lover} да ма́ло даёт\fn{giver}.}
\ordsprog{Der skal skarp \mw{lud} til skurvede hoveder./Med ondt skal ondt fordrives.}
        {
        % slettet По соба́ке и па́лка\fn{stav}.
        Клин\fn{kile} кли́ном вышиба́ют\fn{slår ud}.}
\dui{luft}
\ordsprog{\di{Gud} mildner \mwni{luften} for de klip\-pe\-de \di{får}.}
        {Бог ми́лостив\fn{nådig}.
        % slettet \mitem Одно́ потеря́ешь\fn{mister}, друго́е найдёшь\fn{finder}.
         \mitem Бог не без ми́лости\fn{nåde}, каза́к не без сча́стья\fn{lykke}.}
\dui{lykke}
\ordsprog{Enhver er sin egen \mwni{lykkes} smed.}
        {Ка́ждый кузне́ц\fn{smed} своего́ сча́стья.
        \mitem Всяк своего́ сча́стья кузне́ц.}
\dui{lykke}
\ordsprog{Den ene har \mwni{lykken}, den anden har krykken.}
  {Кому́ пироги́\fn{pirog} да пы́шки\fn{boller}, а кому́ синяки́
  \fn{blå mærker} да ши́шки\fn{buler}.
        % \mitem И то сча́стье, что ино́му вёдро\fn{godt vejr}, ино́му несча́стье.
        }
\dui{lykke}
\ordsprog{\mwni{Lykken} står den kække bi.}
        {Уда́ча\fn{lykken} нахра́п\fn{frækhed, dristighed} лю́бит.
        \mitem Судьба́\fn{skæbnen} покрови́тельствует\fn{beskytter} хра́брым\fn{modig}.
        \mitem Сча́стье сопу́тствует\fn{følger, ledsager} хра́брым.
% slettet        \mitem (Судьба́) сме́лым бог владе́ет\fn{ejer?}.
        \mitem Сме́лость\fn{mod} города́ берёт\fn{indtager}.
% slettet        \mitem Сме́лого пу́ля\fn{kugle} бои́тся.
        \mitem Хра́брому сопу́тствует уда́ча.
}
\dui{lykke}
\ordsprog{\mwni{Lykken} er altid der, hvor du ikke er.}
        {Везде́ хорошо́, где нас нет.}
\dui{lykke}
\ordsprog{\mwni{Lykken} er alle dårers formynder.}
       {
       % slettet Глу́пому сча́стье, у́мному Бог даёт.
      Дураку́ сча́стье.
        \mitem Дураку́ везде́ сча́стье.
        \mitem Дурака́м везёт\fn{have heldet med sig}.
}
\dui{lykke}
\ordsprog{\mwni{Lykken} er af glas.
          (Når den skinner
          klarest, brister den snarest.)}
       {Сча́стье ве́шнее\fn{forårs-} вёдро\fn{vejr}.
        \mitem На сча́стье нет зако́на\fn{lov}.}
\dui{lykke}
\ordsprog{ \mwni{Lykken} kommer, lykken går.}
           {Ле́гче\fn{lettere} сча́стье найти́,
            не́жели\fn{end} удержа́ть\fn{holde på}.}
\dui{lykke}
\ordsprog{\mwni{Lykken} er en underlig trold; hun gækker folk så mangfold.}
{Счастли́вый хо́дит, на клад\fn{skat}
набредёт\fn{\ru{набрести́}=støde på}, а несча́стный\fn{uheldig, ulykkelig}
пойдёт, и гриба́\fn{svamp} не найдёт / и гроша́ не найдёт.}
\dui{lykke}
\ordsprog{\mwni{Lykken} er blind.}
         {Сча́стье без глаз\fn{øjne}.}
\ordsprog{Pris ingen \mw{lykkelig}, før han er \di{død}.}
         {Не говори́ ``гоп''\fn{hop}, пока́ не перепры́гнешь\fn{klækker} (переско́чишь\fn{hopper}).}
\dui{lyst}
\ordsprog{Hver sin \mwni{lyst}.}{
Ка́ждому своё.
}
\dui{lyst}
\ordsprog{\mwni{Lysten} driver værket.}
         {Была́ бы охо́та\fn{vilje}, рабо́та пойдёт\fn{går i gang}.
         \mitem Бы́ло бы жела́ние.
         % Бы́ло бы нелание\fn{???}
         }
\ordsprog{Den kan sagtens \mw{lyve}, som kommer langvejs fra.}
         {Добро́ тому́ врать\fn{lyve}, кто за́ морем быва́л.
        \mitem Хва́лит\fn{praler} чужу́ю\fn{fremmed} страну́, а сам в
              неё ни ного́й\fn{med foden, sat sin fod}.}
\ordsprog{Den der \mw{lyver}, \di{stjæler} også.}
         {Кто лжёт\fn{roser}, тот и крадёт\fn{\ru{красть}=stjæle}.}
\dui{læge}
%\marginpar{Ikke helt forstået.}
\ordsprog{Jo flere \mwni{læger}, jo flere syge.}
             {Та душа́ не жива́, что по лекаря́м
             пошла́.}
\dui{lær}
\ordsprog{\mwni{Lær} selv, før du lærer andre.}
          {Научи́сь сам, пре́жде тем учи́ть други́х.}
\ordsprog{Man skal \mw{lære}, så længe man lever.}
         {Век\fn{århundrede} живи́, век учи́сь.
         \mitem Учи́ться никогда́ не по́здно.
         }
  % slettet
  %\ordsprog{Bor du hos en halt, må du \mw{lære} at halte.}
  %      {С кем поведёшься, от того́ и наберёшься.}
\ordsprog{Liden \di{tue} vælter ofte stort \mw{læs}.}
        {Ма́лый дождь\fn{regn} укроща́ет\fn{tæmmer} больши́е ве́тры\fn{vinde}.
        \mitem ка́пля\fn{dråben} ка́мень\fn{stenen} то́чит\fn{udhuler, slider væk}}
\dui{løgn}
\ordsprog{\mwni{Løgnen} har korte ben.}
          { Ложь\fn{løgn} на тарака́ньих\fn{kakerlak-} но́жках
        \mitem Ложь стои́т до ули́ки\fn{bevis}.}
\ordsprog{En \mw{løgner} må have en god hukommelse.}
         {Лжи́вому на́до па́мятным быть.
        % \mitem Пошли́ тебе́ вруну́\fn{gldags: løgn} бог твёрдую\fn{fast, sikker} па́мять.
        }
\letter{M}
\ordsprog{Fremmed \mw{mad} smager bedst.}
                  {Чужа́я\fn{fremmed} еда́ всегда́
                  ка́жется\fn{forekommer} вкусне́е.}
\ordsprog{Uden \mw{mad} og drikke duer helten ikke. }
               {Без еды́ и питья́ не годи́шься\fn{dur} никуда́
               \mitem У голо́дного\fn{sulten} брю́ха\fn{mave} нет у́ха\fn{øre}.}
\ordsprog{Megen \mw{mad}, megen sygdom.}
       {Больша́я сыта́\fn{honningvand} брю́ху вреди́т\fn{skader}.
        \mitem Где пиры́\fn{middagsselskaber} да чаи́, там и не́мочи\fn{sygdomme}.
        \mitem Сла́дко е́стся, пло́хо спи́тся\fn{søvn}.}
\dui{mad}
\ordsprog{Man skal holde op, når \mwni{maden} smager bedst.}
              {Ешь не доеда́й\fn{spise op}, пей не допива́й\fn{drikke ud}.
        % \mitem Ешь вполсы́та\fn{halvmæt}, не пей до полупья́на\fn{halvfuld}.
        }
\dui{magt}
\ordsprog{\mwni{Magt} går for ret.}
        {Кто сильне́е\fn{stærkere}, тот и праве́е\fn{har mere ret}.
        \mitem Чья сильне́е, та и праве́е.
        \mitem У си́льного всегда́
        беcси́льный\fn{den svage} винова́т\fn{skyldig}.}
% slettet \ordsprog{Man kan ikke se en \mw{mand} længere end til tænderne.}
%         {По пла́тью\fn{beklædning}(одёжке) встреча́ют,
%              а по уму́ провожа́ют\fn{ledsage, tilbringe tiden}.}
%%\ordsprog{Mands vilje er \mw{mands} himmerrig.}{}
\dui{mave}
\ordsprog{\mwni{Maven} bliver mæt før øjnene.}
         {Глаза́ завиду́щие\fn{misundelige}, (ру́ки загребу́щие\fn{begærlig}).
        \mitem Весь сыт\fn{mæt}, а глаза́ голо́дные\fn{sultne}.}
\ordsprog{Fuld \mw{mave} studerer ikke gerne.}
               {Сы́тое\fn{fuld} брю́хо\fn{mave} к уче́нию глу́хо\fn{sløv}.}
  %\ordsprog{Fuld \mw{mave} gør døv for alt.}
  %               {Больша́я сыть брю́ху вреди́т.}
\ordsprog{Man kan ikke på én gang blæse og have \mw{mel} i munden.}
         {Нельзя́ совмести́ть\fn{forbinde} несовмести́мое\fn{uforbindelige}.}
  %\ordsprog{Jeg er et \mw{menneske}, og
  %              intet menneskeligt er mig fremmed.}
  %              {Ни от чего́ челове́ческого не отре́каюшь.}
\ordsprog{Menneske er \mw{menneske} værst.}
                   {Челове́к челове́ку --- волк\fn{ulv}.}
\dui{menneskeligt}
\ordsprog{Det er \mwni{menneskeligt} at fejle.}
        {Челове́ку сво́йственно\fn{særegent} ошиба́ться\fn{fejle}.
        \mitem Все мы гре́шны\fn{syndige}.}
\dui{mester}
\ordsprog{Arbejdet priser \mwni{mesteren}}{
Рабо́та ма́стера хва́лит.
\mitem Тако́в ма́стер, такова́ и рабо́та.
}
\ordsprog{Hold dig til den gyldne \mw{middelvej}.}
               {Сле́дует держа́ться золото́й середи́ны.
               (Избра́ть золоту́ю середи́ну).
               \mitem Серёдка всему́ де́лу ко́рень\fn{rod}.
       % \mitem Де́ло серёдкого кре́пко.
       }
\ordsprog{Man må ofte gøre gode \mw{miner} til slet \di{spil}.}
              {(С)де́лать хоро́шую ми́ну
              при плохо́й игре́.}
\dui{modsætning}
\ordsprog{ \mwni{Modsætninger} mødes.}
           {Кра́йности\fn{modsætninger} (противополо́жности\fn{modsætninger}) схо́дятся.}
\dui{morgenstund}
\ordsprog{Morgenstund har guld i \mw{mund}.}
              {У́тренний\fn{morgen-} час да́рит\fn{\ru{дари́ть}=skænke} зо́лотом нас.
        \mitem Кто ра́но встаёт, тому́ Бог даёт (подаёт).
        \mitem Кто ра́но встаёт, умне́е быва́ет.
        \mitem Нажива́ть\fn{vinde, erhverve}, так ра́ньше встава́ть.
        \mitem Вста́нешь ра́ньше, шагнёшь\fn{tager et par skridt} да́льше.
        % \mitem У́тро ве́чера мудрене́е\fn{klogere}.
        }
\ordsprog{Det er en ussel \mw{mus}, der ej haver uden ét hul.}
              {Худа́ та мышь, кото́рая
              одну́ то́лько лазе́йку\fn{smutvej} зна́ет.}
\ordsprog{Mange gør en \mw{myg} til en elefant.}
                {Не сле́дует из му́хи\fn{flue} слона́.}
\dui{myre}
\ordsprog{En \mwni{myre} har og galde.}
             {И у ку́рицы\fn{høne} се́рдце\fn{hjerte} есть.}
\ordsprog{Man skal ikke græde over spildt \mw{mælk}.}
              {
              Поте́рянного\fn{tabte} не воро́тишь\fn{gøres om}.
              \mitem Слеза́ми\fn{med tårer} го́рю\fn{sorg} не помо́жешь.
              %Сня́вши головы́ по волоса́м не пла́чут.
        \mitem Что с во́зу\fn{læs} упа́ло\fn{faldt}, то пропа́ло.
        \mitem Сде́ланного не воро́тишь.}
\ordsprog{Den, der kommer først til \mw{mølle}, får først malet.}
              {
        Кто пришёл, того́ и помо́л\fn{malede}.
        \mitem  Кто перве́е, тот и праве́е.
        \mitem Кто ра́но встаёт, тому́ Бог даёт.
        \mitem Чей черёд\fn{tur}, то и берёт\fn{vælger}.}
\ordsprog{Alting med \mw{måde}.}
                 {Всё в ме́ру\fn{måde}.
        \mitem Всему́ есть ме́ра.
       % \mitem Арши́н\fn{mål (71 cm)} на сукно́\fn{klæde}, кувши́н\fn{kande} на вино́.
       }
\dui{mådehold}
\ordsprog{\mwni{Mådehold} er det bedste lægemiddel.}
        {Уме́ренность\fn{mådehold, moderation} --- мать здоро́вья.
        \mitem Живи́ про́сто, проживёшь до́ ста\fn{op til hundrede}.}
\letter{N}
\ordsprog{God \mw{nabo} er bedre end broder i anden by.}
              {Бли́зкий сосе́д лу́чше да́льней родни́\fn{familie}.}
\ordsprog{Elsk din \mw{nabo}, men riv ikke gærdet imellem jer ned.}
               {С сосе́дом дружи́, а забо́р\fn{gærde} чини́\fn{reparer, vedligehold}.}
\ordsprog{En nar kan spørge oм mere end ti  vise kan svare på.}
{
На вса́кого дурака́, у́ма не напасёшься\fn{\ru{напасти́}=angribe}.
}
\ordsprog{Det er bedre at gå til sengs
              uden \mw{nadver} (aftensmad) end at stå op med gæld.}
              {Лу́чше без у́жина ложи́ться,
              чем с долга́ми встава́ть.}
\dui{nat}
\ordsprog{\mwni{Natten} bringer råd.}
                        {У́тро ве́чера мудрене́е\fn{visere}.}
\ordsprog{Noget for \mw{noget}.}{Услу́га за услу́гу.}
\dui{nyttig}
\ordsprog{Den har fået alle stemmer, som har
              forstået at forene det \mwni{nyttige}
              med det behagelige.}
              {Соедини́ть\fn{kombinere} прия́тное с поле́зным\fn{nyttig}.
              \mitem Меша́й де́ло\fn{forretning} с безде́льем\fn{lediggang},
              проживёшь век\fn{en evighed, lang tid} с весе́льем\fn{munterhed}.}
\ordsprog{Man skal sætte tæring efter \mw{næring}.}
         {По одёжке\fn{klæder} протя́гивай\fn{strække} но́жки\fn{fødderne}.}
\dui{nærmest}
\ordsprog{Enhver er sig selv \di{nærmest}.}
                 {Своя́ руба́шка\fn{skjorte} бли́же к те́лу\fn{krop}.
        \mitem Всяк про себя́, а Госпо́дь\fn{Herren} про всех.
        \mitem Род да пле́мя\fn{stamme, slægt, folk} бли́зки, а свой рот\fn{mund} бли́же.
        \mitem Вся семья́ своя́, да всяк лю́бит себя́.
        \mitem Е́сли не я себе́, то кто помо́жет мне?}
\dui{nærved}
\ordsprog{\mwni{Nærved} skyder ingen hare.}
        {Чуть\fn{næsten, lille smule} --- чуть не счита́ется\fn{tæller}.
 % slettet       \mitem Про́мах\fn{forbier} есть про́мох.
 % slettet        \mitem Ви́дит о́ко\fn{øje}, да зуб\fn{tand} неймёт.
 % slettet       \mitem От аво́ся\fn{måske, forhåbentlig} добра́ не жди.
        }
\dui{nød}
\ordsprog{I \mwni{nøden} skal man kende sine venner.}
            {Друзья́ позна́ются в беде́\fn{ulykke}.
        \mitem Без беды́ дру́га не узна́ешь.}
\dui{nød}
\ordsprog{\mwni{Nød} bryder alle love.}
       {Нужда́ зако́н\fn{lov} лома́ет\fn{bryder}.
        \mitem Нужда́ желе́зо лома́ет.
        \mitem Нужда́ заста́вит\fn{tvinger til} пойти́ на всё.
% slettet        \mitem По нужде́ поп\fn{paven} ест и боб\fn{bønner}.
        }
\dui{nød}
\ordsprog{Når \mwni{nøden} er størst, er hjælpen nærmest.}
        {Чем нужда́ бо́льше (сильне́й), тем бли́же по́мощь.
      %  \mitem Чем бли́же беда́, тем бо́льше ума́\fn{forstand}.
        \mitem Тьма\fn{mørket} сильне́й пе́ред рассве́том\fn{morgengry}.
      % slettet  \mitem Нет ху́де\fn{ru{ху́до}=onde} без добра́.
        }
\dui{nødvendighed}
\ordsprog{Man må ofte gøre en dyd af \mwni{nødvendigheden}.}
 {
 % (С)делать из нужды́ доброде́тель.
 Когда́ нет раба́\fn{slave}, и сам по дрова́\fn{brænde}.}
\dui{nød}
\ordsprog{Nød lærer \mw{nøgen} kvinde at \di{spinde}.}
              {Нужда́ нау́чит кузнеца́\fn{smed} сапоги́\fn{støvler} тача́ть\fn{касте, sy med kastesøm}.
        \mitem Нужда́ нау́чит бо́гу моли́ться\fn{at bede}.
        \mitem Голь\fn{fattigfolk} на вы́думки\fn{hittepåsomhed} хитра́\fn{listig, snu}.}
\letter{O}
\ordsprog{Sig mig, hvem du \mw{omgås}, så
         skal jeg sige dig, hvem du er.}
              {Скажи́ с кем ты знако́м, и я
              скажу́, кто ты тако́в.
              \mitem Скажи́ кто твой друг, и я
              скажу́, кто ты.}
\ordsprog{Med \mw{ondt} skal ondt fordrives.}
              {Клин\fn{kile} кли́ном вышиба́ют.
        \mitem В чём грех\fn{synd}, в том и спасе́нье\fn{frelse}.
        % slettet \mitem Чем уши́бся, тем и лечи́шь.
        }
\dui{opsætte}
\ordsprog{\mwni{Opsæt} ikke til i morgen, hvad du kan gøre i dag.}
             {Не оставла́й на за́втра то что мо́жно сдела́ть сего́дня
             \mitem Одно́ ны́нчe\fn{nu, i dag} лу́чше двух за́втра
             % \mitem Поцелу́й меня́ сего́дня, а я тебя́ за́втра.
             }
\ordsprog{Et \mw{ord} er et ord og en \di{mand} er en mand.}
         {Да́вши\fn{havende givet} сло́во, держи́сь\fn{hold det}.}
\ordsprog{Det \mw{ord} du har sagt, prøv om du kan tage det igen.}
              {Ска́заннoго не воро́тишь\fn{vender}.
        \mitem Что напи́сан перо́м\fn{med pen}, не вы́рубишь топоро́м\fn{med økse}.
        \mitem Сло́во не воробе́й\fn{spurv}: вы́летит\fn{flyver ud},
               не пойма́ешь.}
\dui{ord}
\ordsprog{\mwni{Ord} og \di{gerning} er to ting.}
                     {Ско́ро \ri{сло́во} ска́зывается,
                     а не ско́ро
                     \ri{де́ло} де́лaется.
        \mitem Мно́го слов, а ма́ло де́ла.}
\dui{orden}
\ordsprog{\mwni{Orden} er det halve \di{arbejde}.}
         {Поря́док --- душа́\fn{sjæl} вся́кого де́ла.}
\ordsprog{Enhver har sin \mw{orm}.}
         {У ка́ждого свои́ причу́ды\fn{særheder, luner}.}
\dui{orm}
\ordsprog{Selv \mwni{ormen} krymper sig i døden.}
               {И у \ri{ку́рицы} \ri{се́рдце} есть.}
\dui{ormeføde}
\ordsprog{Vi skal alle engang blive til \mwni{ormeføde}.}
         {Мы все когда́-нибудь пойдём на корм\fn{føde} червя́м\fn{orme}.}
\letter{P}
\dui{papir}
\ordsprog{\mwni{Papiret} er tålmodigt.}
         {Бума́га всё (с)те́рпит.}
\dui{part}
\ordsprog{Man må høre begge \mwni{parter}.}
         {Ну́жнo вы́слушать о́бе сто́роны.}
\ordsprog{Penge avler \mw{penge}.}
         {
         %Де́ньги иду́т к бога́тому\fn{den rige}.
         Де́ньги ро́дят де́ньги.
         }
\ordsprog{Hvor \mw{penge} er, kommer penge til.}
         {Де́ньги к деньга́м.}
\dui{penge}
\ordsprog{Pemge går hurtigt.}{Де́ньги про́сто та́ют.}
%\marginpar{samme betydning?}
%\ordsprog{Stjålne \mw{penge} varer ikke længe.}
%         {Чужи́мэ\fn{fremmede} бога́т не бу́дешь.}
\dui{penge}
\ordsprog{\mwni{Penge} gør manden.}
         {За свой грош везде́\fn{overalt} хоро́ш.}
\dui{penge}
\ordsprog{\mwni{Penge} skaber ikke lykken.}
         {Не в деньга́х сча́стье.
        \mitem Бо́льше де́нег, бо́льше хлопо́т\fn{besvær}.}
%\dui{penge}
%\ordsprog{\mwni{Penge} er runde, fordi de skal trille.}
%       {Де́нььги подплыва́ют\fn{svømmer}.
%        \mitem Де́ньги родя́т\fn{avler} де́ньги.}
\ordsprog{For \mw{penge} får man alt.}
         {Де́ньги всё мо́гут.}
\ordsprog{Penge lugter ikke}{Де́ньги не па́хнут.}
\dui{perle}
\ordsprog{Man skal ikke kaste \mwni{perler} for svin.}
         {Не мечи́те\fn{kaste} би́сер\fn{perler} пе́ред сви́ньями.}
\ordsprog{Man kan ikke få både i \mw{pose} og sæk.}
              {
    Оди́н пиро́г два ра́за не съешь.
   \mitem Нельзя́ де́лать одновре́менно две взаимоисключа́ющие\fn{indbyrdes uforenelige} ве́щи.
%              За двумя́ за́йцами\fn{harer} пого́нушься\fn{jager}, ни одного́ не пойма́ешь\fn{fanger}.
              }
\ordsprog{Alting har sin pris}{За всё (в жи́зни) на́до плати́ть\fn{betale}.}
\ordsprog{En \mw{profet} er aldrig agtet i sit fædreland.}
             {Нет проро́ка\fn{profet} в своём оте́честве\fn{fædreland}.
        \mitem На свое́й земле́\fn{land} никто́ проро́ком не быва́ет.}
\dui{præst}
\ordsprog{Når det regner på \mwni{præsten}, drypper det på degnen.}
              {Свяще́нник зарабо́тал\fn{tjente} и
              пономарю́\fn{kirketjener} перепа́ло\fn{tilfaldt}.}
\dui{præst}
\ordsprog{Det går ikke altid så skidt, som \mwni{præsten} prædiker.}{Не так стра́шен\fn{frygtelig} чёрт\fn{djævlen}, как его́ малю́ют\fn{man maler}.}
\ordsprog{Man kan let give af en andens \mw{pung}.}
              {Из чужо́го карма́на легко́ плати́ть\fn{betale}.}
%\ordsprog{Man sætter \mw{pølse} for og pølse bag, pølsen bevarer dog sin smag.} {Куда́\fn{ikke forstået} ни кинь, всё (везде́/всю́ду) клин.}
\letter{R}
\ordsprog{Den ene \mw{ravn} hugger ikke øjet ud på den anden.}
              {Во́рон во́рону глаз (гла́за) не вы́клюет\fn{hakke ud. \ru{клева́ть}=hakke}.
        \mitem Вор во́ра не вы́даст\fn{\ru{вы́дать}=stikke, udlevere}.}
\ordsprog{Man skal ikke nævne \mw{reb} i den hængtes hus.}
              {В до́ме пове́шенного\fn{hængte} не говоря́т о верёвке\fn{reb}.}
\ordsprog{Ingen \mw{regel} uden undtagelse.}
                {Нет пра́вила\fn{regler} без исключе́ния.}
\ordsprog{Efter \mw{regn} kommer solskin.}
                {По́сле до́ждика (дождя́) и вёдро\fn{smukt vejr} бу́дет (бу́дет со́лнышко\fn{solskin}).}
\ordsprog{Den, der gør \mw{regning} uden vært, må regne om igen.}
         {% slettet Гру́бо посчита́ться.
         Без меня́ меня́ жени́ли\fn{giftede}.
        % \mitem Дейтвовать, не прининма́я в расчёт обстоя́тльст?????
         }
\ordsprog{Godt \mw{regnskab} gør (giver) godt \di{venskab}.}
               {Чаще счёт, кре́пче\fn{stærkere} дру́жба.
%        \mitem Де́ньги счёт\fn{regnskab, afregning} лю́бят.
%        \mitem Угово́р\fn{enighed} доро́же (лу́чше) де́нег.
%        \mitem Де́нежка\fn{lille skilling} счёт лю́бит.
}
\dui{relativt}
\ordsprog{Alt er \mwni{relativt}.}
        {Всё относи́тельно.}
\ordsprog{Man skærer ofte bred \mw{rem} af en andens \di{hud}.}
        {Из чужо́й спины́ реме́шки\fn{\ru{ремешо́к} = lille rem} крои́ть.
        % \mitem Легко́ на чужи́е грехи́ е́хать.
        \mitem Бе́лые\fn{hvide} ру́чки\fn{håndtag, dørgreb, hank} чужи́е\fn{andres} труды́\fn{arbejde} лю́бят.}
\dui{ren}
\ordsprog{For den \mwni{rene} er alting rent.}
       {Чисто́му всё чи́сто.
        \mitem Хоро́шему всё хорошо́.}
% slettet \ordsprog{Højeste \mw{ret} er ofte uret.}{}
\ordsprog{Mangen \mw{rider} ikke den dag de sadler.}
         {Ра́но оседла́ли\fn{sadlede, besteg hest}, да по́здно
                 поскака́ли\fn{satte i galop}.
%          \mitem Он всё де́лает в час по ча́йной ло́жке\fn{teske}.
}
\ordsprog{Det er svært for den rige at komme i Himmerigs Rige.}{
Тру́дно бога́тому войти́ в Ца́рство Небе́сное.
}
\ordsprog{Alle veje fører til \mw{Rom}.}{Все доро́ги
               веду́т в Рим (в Москву́).}
\ordsprog{Man kan spørge sig frem til \mw{Rom}.}
              {Язы́к до Ки́ева доведёт\fn{vil bringe}.}
\ordsprog{\mw{Rom} blev ikke bygget på en dag.}
         {Не в оди́н день Москва́ стро́илась\fn{blev bygget}.
        \mitem И Москва́ не сра́зу стро́илась.}
\dui{rose}
\ordsprog{Der er ingen \mwni{roser} uden torne.}
              {Нет ро́зы без шипо́в\fn{torne}.}
\dui{rotte}
\ordsprog{\mwni{Rotterne} forlader den synkende skude.}
            {Кры́сы\fn{rotter} бегу́т с то́нущего\fn{sunkne} корабля́\fn{skib}.}
\dui{rygte}
\ordsprog{\mwni{Rygtet} vokser, mens det løber.}
         {До́брая сла́ва\fn{omdømme/rygte} лежи́т, а дурна́я
         молва́\fn{rygte} далеко́ бежи́т.}
\dui{rygte}
\ordsprog{\mwni{Rygtet} lever, når manden er død.}
                        {От молвы́ не уйдёшь.}
\ordsprog{Godt \mw{rygte} er det bedste arve\-gods.}
               {До́брая сла́ва лу́чше зо́лота.
      % slettet  \mitem До́брая сла́ва в углу́(за печкой\fn{pejsen-ovnen}) сиди́т, худа́я сла́ва по доро́жке\fn{(gl.) sti, vejsti} бежи́т.
     }
\ordsprog{Med \mw{ræv} fanger man ræv.}
              {Клин\fn{kile} кли́ном вышиба́ют\fn{puffer, støder}.}
\ordsprog{En gammel \mw{ræv} er ikke let at fange.}
         {Ста́рую лиси́цу\fn{ræv} не тра́вят\fn{jage, forgifter} молоды́ми соба́ками.
        \mitem Ста́рого воробья́\fn{spurv} на мяки́не\fn{avner} не пойма́ешь
        (не обма́нешь\fn{narrer}).
      %  \mitem Ста́рый конь\fn{hest} ми́мо\fn{forbi} не сту́пит\fn{træder}.
      }
\dui{ræv}
\ordsprog{De er sure!, sagde \mwni{ræven} om \di{rønnebær}rene.}
             {Зе́лен виногра́д\fn{vindruer}!
       % slettet \mitem ви́дит о́ко (глаз), да зуб неймёт.
       }
\dui{ræv}
\ordsprog{Man skal ikke lade ræven vogte gæs.}{Э́то всё равно́ поста́вить лису́ стере́чь гусе́й}
\ordsprog{I dag \mw{rød}, i morgen død.}
            {Сего́дня живо́й, за́втра свято́й\fn{helgen}.
        \mitem Сего́дня венча́лся\fn{blev kronet/giftede sig}, а за́втра сконча́лся\fn{døde}.
        \mitem Сего́дня в порфи́ре\fn{purpurfarvet sten, porfyr}, а за́втра в моги́ле\fn{grav}.}
\ordsprog{Der går ikke \mw{røg} af en brand, uden
              at der er ild i den.}
              {Нет ды́ма без огня́.
        \mitem Ды́ма без огня́ не быва́ет.}
\ordsprog{Man får altid \mw{råd} nok, når man ingen behøver.}
         {Да́ли оре́хов\fn{nødder} бе́лке\fn{egern}, когда́ зубо́в\fn{tænder} не ста́ло\fn{ikke var mere}.}
\ordsprog{Et godt \mw{råd} er bedre end penge.}
             {Хоро́ший сове́т доро́же зо́лота.}
\dui{råde}
\dui{mennesket}
\dui{Gud}
\ordsprog{Mennesket spår, Gud rå'r (\mwni{råder}).}
          {Челове́к предполага́ет\fn{antager, forventer}, Бог рас\-полага́\-ет\fn{råder}.}
\ordsprog{Når alle vil \mw{råde}, giver det ingen til både (gavn).}
         {Мно́гие жале́ют\fn{beklager, undskylder}, да не́кому помо́чь.}
\ordsprog{Det er bedre at være \mw{rådvild} end \di{husvild}.}
         {Лу́чше не знать то́чно, что де́лать,
         чем не знать, куда́ идти́.}
\letter{S}
\dui{salt}
\ordsprog{\mwni{Salt} råder for maden alt.}
         {Без со́ли стол\fn{bord} криво́й\fn{skævt}.}
\ordsprog{Efter en \mw{samler} kommer en spreder.}
         {Оте́ц накопи́л\fn{akkumulerer}, а сын раструси́л\fn{spredte, strøede}.
         \mitem Отцы́ нажива́ют\fn{skraber sammen, tjener}, де́ти прожива́ют\fn{sætter over styr, bruger til at leve}.}
\ordsprog{En god \mw{samvittighed} er den bedste \di{hovedpude}.}
         {У кого́ со́весть\fn{samvittighed} чиста́, тот спит споко́йно\fn{hyggeligt, trygt}.
         \mitem У кого́ со́весть чиста́, у того́ поду́шка\fn{pude} не ве́ртится\fn{drejer rundt}.
        % slettet \mitem До́брая со́весть не бои́тся клеве́т\fn{sladder}.
        }
\dui{sandhed}
\ordsprog{\mwni{Sandhed} er rigere end \di{loven}.}
         {Пра́вда доро́же зо́лота\fn{samme betydning?}.
        \mitem Варва́ра\fn{navn?} мне тётка\fn{tante}, а пра́вда мне сестра́ (мать).}
\dui{sandhed}
\ordsprog{\mwni{Sandheden} slår ikke til for den, der har munddiarré.}
         {Пра́вда не речи́ста\fn{godt skåret for tungebåndet, snakkesalig}.
        \mitem На пра́вду ма́ло слов.}
\dui{sandhed}
\ordsprog{\mwni{Sandheden} er altid ilde hørt.}
         {Пра́вда глаза́ ко́лет\fn{\ru{коло́ть} stikke, prikke}.
        \mitem  Пра́вда уши́ дерёт\fn{\ru{драть} = flå}.
        \mitem Пра́вду говори́ть, дру́жбу\fn{venskab} теря́ть\fn{miste}.
        \mitem Пра́вде нигде́ нет ме́ста.}
\dui{sandhed}
\ordsprog{\mwni{Sandhed} består, løgn forgår.}
         {Всё ми́нется\fn{forgår}, одна́ пра́вда оста́нется\fn{bliver tilbage, består}.}
\ordsprog{Sjældent fanger sovende mand \mbox{\mw{sejr}}.}
         {До́лго спать, с до́лгом\fn{med gæld} встать.}
\ordsprog{Slet \mw{selskab} fordærver gode sæder.}
               {
               % slettet Злы́е\fn{dårlige, onde} бесе́ды\fn{samtaler}
              % тлят\fn{\ru{тлить} = ødelægge} обы́чаи\fn{vaner}
              % благи́е\fn{gode}.
        Дурны́е приме́ры зарази́тельны\fn{smitsomme}.
        \mitem С худы́м поведёшься\fn{\ru{повести́сь} = indlade sig med?}, худо́е переймёшь\fn{\ru{переня́ть} = overtage}.
        \mitem С кем поведёшься\fn{bevæger dig}, от того́ и наберёшься\fn{stimle sammen.}}
\dui{selvgjort}
\ordsprog{\mwni{Selvgjort} er \di{velgjort}.}
         {
        Лу́чше всего́ то, что сам сде́лаешь.
        \mitem Не верь по́вару\fn{kok}, иди́ сам по́ воду.
        \mitem Полага́йся\fn{stol på} на самого́ себя́.
        % slettet \mitem Своя́ рука́ влады́ка\fn{hersker}.
        % slettet \mitem Своя́ во́ля влады́ка.
}
\dui{selvros}
\ordsprog{\mwni{Selvros} \di{stinker}.}
%    slettet    {Самовосхвале́ние\fn{selvglorificering}отврати́тельно\fn{afskyeligt}\fn{Note jeg ikke forstår \ru{похвальта́}}.
        {Сам себя́ ни хвали́\fn{ros}, ни хули́\fn{hån}.
        \mitem Пога́на\fn{uren, rådden} та де́вка\fn{pige, tøs}, что сама́ себя́ хвали́т.
        \mitem Саморекла́ма\fn{selvpromovering} --- ещё не рекоменда́ция\fn{anbefalet}.}
\ordsprog{Bedre \mw{sent} end aldrig.}
                {Лу́чше по́здно, чем никогда́.}

\ordsprog{Som de \di{gamle} \mw{sjunge}, så \di{kvidre} de
\di{unge}.}
        {Каковы́ дя́дьки\fn{onkler}, таковы́ и дитя́тки\fn{børn}.
        \mitem Како́в поп\fn{præst}, тако́в и прихо́д\fn{menighed}.
        \mitem За что ба́тька\fn{fader}, за то и де́ти.}
\ordsprog{En \di{sund} \mw{sjæl} i et sundt \di{legeme}.}
        {В здоро́вом те́ле\fn{krop} --- здоро́вый дух\fn{ånd, sjæl}.}
\ordsprog{Af \mw{skade} bliver man klog, men sjældent rig.}
             {От беды́\fn{ulykke} умне́ют, да ре́дко богате́ют.}
    %  slettet  \mitem Беды́ науча́ют челове́ка и му́дрости.
    %  slettet  \mitem Обжёгшись на молоке́, ста́нешь дуть и на́ воду.}
\ordsprog{Jo mere man rører ved \mw{skarn}, desto mere stinker det.}
             {Гря́зью\fn{mudder} игра́ть --- ру́ки мара́ть\fn{gøre beskidt}.
        \mitem Не трись\fn{\ru{тереться}=gnide, tørre, frottere sig} во́зле\fn{ved siden af} са́жи\fn{sod},
             сам замaра́ешься\fn{bliver snavset}.}
\ordsprog{Sket er \mw{sket}.}
               {Что случи́лось, то случи́лось.
               \mitem Что с во́за\fn{\ru{воз}=vognlæs} упа́ло, то пропа́ло.}
\ordsprog{Når \mw{skidt} kommer til ære, ved det ikke hvordan
              det vil være.}
              {Посади́\fn{sæt} свинью́ за стол\fn{bord}, она́ и но́ги на стол\fn{benene på bordet}.}
%\dui{skidt}
% slettet\ordsprog{\mwni{Skidt} renser tarmene, skader ikke.}
%         {Бо́льше гря́зи, толще морда.}
\dui{skik}
\ordsprog{\mwni{Skik} følge eller land fly.}
         {В чужо́м\fn{fremmed} до́ме не ука́зывают\fn{anviser}.
        \mitem В чужо́й монасты́рь\fn{kloster} со свои́м уста́вом\fn{reglement} не е́здят (не хо́дят).
% slettet        \mitem С волка́ми\fn{ulve} жить, по во́лчьи\fn{ulvemaner} выть\fn{hyle}.
}
\dui{skik}
\ordsprog{Hvert land har sine \mwni{skikke}.}
                {Ско́лько сёл\fn{landsbyer}, сто́лько вер\fn{overbevisninger}.
        \mitem Ско́лько ме́льников\fn{møllere}, сто́лько мер\fn{mål}.
        \mitem Что го́род; то но́ров\fn{særpræg??}, что дере́вня\fn{landsby}, то обы́чай\fn{vannligt}.}
\ordsprog{Den, der er slået til \mw{skilling}, bliver aldrig til
               en daler.}
               {Осла́\fn{æsel} хоть в Пари́ж\fn{Paris}, всё бу́дет рыж\fn{rødhåret}.
        \mitem Роди́лся неу́мный\fn{uden forstand}, и умрёшь дурако́м\fn{tåbe}.}
\ordsprog{Den, der ikke vil have en \mw{skilling}, får aldrig to. }
         {Без копе́йки рубль не живёт (рубля́ не быва́ет).}
\dui{skinnet}
\ordsprog{\mwni{Skinnet} bedrager.}
         {Вне́шность\fn{udseende} обма́нчива\fn{bedrager}.
        \mitem Не всё, что се́ро\fn{gråt}, волк\fn{ulv}.
        \mitem Приме́ты\fn{tegn, varsler} в све́те\fn{lyset} ча́сто лгут\fn{lyver}.
        \mitem Не вся́кий плут\fn{slyngel}, кто ви́дом худ\fn{tynd}.}
\dui{sko}
\ordsprog{Den ved bedst, som har \mwni{skoen} på, hvor den trykker.}
              {
              % slettet Что у кого́ боли́т, тот о том и говори́т.
        Ка́ждый зна́ет, в чём его́ сла́бое ме́сто\fn{svage punkt} (недоста́ток\fn{mangel}/пробле́ма).}
\dui{skomager}
\ordsprog{\mwni{Skomager}, bliv ved din læst.}
         {Знай сверчо́к\fn{fårekylling}, свой шесто́к\fn{arne}.
        \mitem Всяк сверчо́к, знай свой шесто́к.}
\dui{skomager}
\dui{smed}
\ordsprog{\mwni{Skomagerens} kone (børn) og smedens hest har
                             gerne de dårligste sko.}
         {Портно́й\fn{skrædder} без порто́к\fn{bukser}, сапо́жник\fn{skomager} без сапо́г\fn{sko}.}
\dui{skov}
\dui{træ}
\ordsprog{Ofte kan man ikke se \mwni{skoven} for træer.}
         {Из-за дере́вьев ле́са не ви́дит.
        \mitem За дере́вьями ле́са не ви́дит.}
\dui{skov}
\ordsprog{Hvad skal han i \mwni{skoven}, der ræddes for hver busk.}
         {Волко́в\fn{ulve} боя́ться --- в лес не ходи́ть.
        \mitem Во́лка боя́ться, быть без грибо́в\fn{svampe}.}
\ordsprog{Det er det første \mw{skridt}, der koster.}
         {Тру́ден то́лько пе́рвый шаг.
         \mitem Лиха́\fn{ond, slem, slet} беда́\fn{ulykke} нача́ло\fn{begyndelse}.
        \mitem Вся́кое нача́ло тру́дно.}
         %%%%%%%%%%% her
\ordsprog{Man kan ikke springe over sin egen \mw{skygge}.}
              {Ло́шадь быстра́, да не уйдёт\fn{gå fra} от хвоста́\fn{hale}.
        \mitem Себя́ не переплю́нешь\fn{\ru{перепло́ныть}=spytte længere, overgå (\ru{плева́ть} = spytte)}.}
\dui{skynde}
\ordsprog{\mwni{Skynd} dig langsomt.}
         {Спеши́\fn{hav travlt, skynd dig} ме́дленно.
         \mitem Торопи́ться\fn{skynder sig, haster} ме́дленно.
  % slettet       \mitem И ре́дко\fn{sjældent} шага́ет\fn{tager et skridt}, да твёрдо\fn{fast} ступа́ет\fn{træder}.
  }
\ordsprog{Ingen kan undgå sin \mw{skæbne}.}
                {
                Никто́ не уйдёт от свое́й судьбы́\fn{skæbne}.
             \mitem От судьбы́ не уйдёшь.
        \mitem От своего́ ро́ка(ро́ку)\fn{\ru{рок} = skæbne} никто́ не уйдёт.
        \mitem Нельзя боро́ться\fn{kæmpe mod} с судьбо́й.
        }
\ordsprog{Hvi ser du \mw{skæven}, som er i din broders øje, men
              bjælken i dit eget bliver du ikke var.}
              {В чужо́м глазу́ сучо́к\fn{gren/knast}(сори́нку\fn{støvkorn}) ви́дим, а в своём и бревна́\fn{\ru{бревно́}=bjælke}
              не замеча́ем.}
\dui{smag}
\ordsprog{\mwni{Smag} og behag er forskellig.}
                      {О вку́сах\fn{smag} не спо́рят.
        \mitem На вкус и на цвет\fn{farve}, това́рища\fn{kammerat} нет.
        \mitem У ка́ждого свой вкус.}
\dui{smuler}
\ordsprog{\mwni{Smuler} er også brød.}
         {Не пренебрега́й\fn{være nedladende, ringeagte, forsømme} ма́лым.}
\dui{smøre}
\ordsprog{Den, der \mwni{smører} godt, kører godt.}
         {Не подма́жешь\fn{smører}, не пое́дешь.
        \mitem Суха́я\fn{tør} ло́жка\fn{ske} рот\fn{mund} дерёт\fn{\ru{драть}=flå, flænse, rive itu}.}
\ordsprog{Hver \mw{so} synes bedst om sine grise.}
               {Вся́кому своё ми́ло.
               \mitem Всяк кули́к\fn{ryle} своё боло́то\fn{sump, mose} хва́лит\fn{roser}.}
\dui{sol}
\ordsprog{Når man taler om \mwni{solen}, så skinner den.}
              {О во́лке\fn{ulv} речь, а волк навстре́чь.
        \mitem Лёгок\fn{let} на поми́не\fn{erindring, ihukommelse}.}
\dui{sol}
\ordsprog{Intet nyt under \mwni{solen}.}
         {Нет ничего́ но́вого под со́лнцем.
         \mitem Ничто́ не но́во под луно́й\fn{månen}.}
\dui{sol}
\ordsprog{\mwni{Solen} har også sine pletter.}
         {И на со́лнце есть пя́тна\fn{pletter}.}
% slettet for nu \dui{sol}
% \ordsprog{\mwni{Solen} går op over onde og gode.}
%          {И за гора́ми\fn{bjergene} живу́т\fn{bor} лю́ди.}
\dui{sol}
\ordsprog{Lad ikke \mwni{solen} gå ned over din vrede.}
         {Со́лнце не зайдёт во гне́ве\fn{vrede} ва́шем.}
\dui{sol}
\ordsprog{Ingen kender dagen før \mwni{solen} går ned.}
         {Хвали́\fn{ros} у́тро днём, а день ве́чером.}
\ordsprog{Det er en dårlig \mw{soldat},
              der ikke tror at kunne blive general.}
         {Плох тот солда́т, кото́рый не
          мечта́ет (не наде́ется) стать генера́лом.}
\dui{sorg}
\ordsprog{\mwni{Sorg} og \di{glæde}, de vandre til hobe.}
         {Не узна́в го́ря\fn{sorg, ulykke}, не узна́ешь и ра́дости.}
\ordsprog{Én \mw{sorg} kommer sjældent alene.}
         {Беда́ не прихо́дит одна́.}
\dui{spare}
\ordsprog{\mwni{Spar}!, så har du.}
         {Береги́\fn{spar} де́нежку про чёрный\fn{sort, mørk} день.}
\dui{spare}
\ordsprog{Hvad der er \mwni{sparet}, er tjent.}
         {Не истра́тил\fn{brugte}, всё равно́\fn{det samme som} что зарабо́тал\fn{tjente}.
         \mitem Бережли́вость\fn{omhyggelighed, sparsommelighed} лу́чше бога́тства\fn{rigdom}.}
\dui{sparke}
\ordsprog{Man \mwni{sparker} ikke på liggende mand.}
              {Лежа́щего не бьют\fn{slår}.}
\ordsprog{Uheld i \mw{spil}, held i kærlighed.}
         {Несчастли́в в игре́, так счастли́в в любви́.
         \mitem Ему́ везёт\fn{lykkes med} в  игре́, ему́ не везёт в любви́.}
%\ordsprog{Man skal holde op med at \mw{spise}, når maden smager bedst.}{}
\dui{spot}
\ordsprog{\mwni{Spot} og skade følges ad.}
         {Свали́сь\fn{segnede, faldt om} то́лько с ног,
          а за тычка́ми\fn{\ru{тычо́к}=stav} де́ло не ста́нет\fn{standser}.
        \mitem Кого́ бьют\fn{slås}, того́ и браня́т\fn{\ru{брани́ть}=skælde ud}.}
\dui{spørge}
\ordsprog{Som man \mwni{spørger}, får man svar.}
              {Како́в вопро́с, тако́в отве́т.}
%\ordsprog{Det er slemt at støde sig to gange på samme \mw{sten}.}
%              {На один гра́би не наступают дважды.}
\dui{sten}
\ordsprog{Vand udhuler \mwni{stenen}.}
         {Вода́ (и) ка́мень то́чит\fn{udhule, hvæsse, slide op}.}
\ordsprog{Rullende \mw{sten} samler ikke mos.}
         {На одно́м ме́сте (и) ка́мень мхом\fn{\ru{мох}=mos}
         обраста́ет\fn{gror til med}.}
\ordsprog{Man skal ikke kaste med \mw{sten}, når man bor
              i et glashus.}
       {Нельзя́ кида́ться\fn{kastе} камня́ми , когда живёшь
            в стекля́нном до́ме.
       \mitem Не кида́й камня́ми , когда живёшь
            в стекля́нном до́ме.
        \mitem Не броса́й\fn{kast} камне́й\fn{med sten} в чужо́й огоро́д\fn{have}.
        \mitem Не суди́\fn{døm} друго́го за то, в чём сам не безгре́шен\fn{uden synd}.
        \mitem Други́х не суди́, на себя́ погляди́\fn{betragt}.}
\dui{struds}
\ordsprog{\mwni{Strudsen} gemmer sit hoved under vingen.}
{Стра́ус пря́чет\fn{\ru{пря́тать} = gemme, skjule} го́лову под крыло́\fn{vinge}.}
\ordsprog{Efter \mw{storm} kommer stille.}
         {По́сле дождя́\fn{regn} вёдро\fn{godt vejr} наступа́ет\fn{bryder frem}.
         % slettet \mitem Зати́шье пе́ред бу́рей
         }
\dui{stud}
\ordsprog{Den, der ager med \mwni{stude}, kommer også med.}
         {
         И ти́хий воз\fn{(vogn)læs} бу́дет на горе́.
         \mitem Ти́ше\fn{\ru{ти́хий} = stille, stilfærdig} е́дешь, да́льше бу́дешь.
         %\mitem Сперва́\fn{fra begyndelsen} не пры́тко\fn{springsk, urolig}, а там поти́ше\fn{roligt??}.
         }
\dui{stud}
\ordsprog{Gamle \mwni{stude} har hårde horn.}
         {Стар козёл\fn{gedebuk}, да кре́пкие рога́.}
\ordsprog{
        Det er slemt at \mw{støde} sig to gange på samme \di{sten}.}
{На одни́ гра́бли\fn{rive} два́жды\fn{to gange} не наступа́ют\fn{træder på}.}
\ordsprog{Stor \mw{ståhej} for ingenting.}
        {Мно́го шу́ма\fn{larm} из ничего́.
        \mitem Мно́го шу́му, ма́ло то́лку\fn{mening, sund fornuft}.
        \mitem Мно́го гро́му\fn{torden}, а ма́ло де́ла.
        \mitem Есть что слу́шать\fn{høre}, да не́чего\fn{ingen grund til} ку́шать\fn{spise??}.
        \mitem Мно́го гро́му по пусто́му\fn{ingenting}.
        \mitem За всё бра́ться\fn{give sig i kast med}, ничего́ не сде́лать.}
\dui{sult}
\ordsprog{\mwni{Sult} er den bedste kok.}
        {Го́лод лу́чший по́вар.
        % \mitem Где го́лодно, там и хо́лодно.
        }
\dui{sult}
\ordsprog{\mwni{Sult} er et hvast sværd i grådig mave.}
         {Голо́дная ку́рица\fn{høne} ви́дит про́со\fn{hirse} во сне\fn{søvn}.
        \mitem У голо́дного всё хлеб на уме́\fn{forstand, hoved}.
        \mitem Голо́днои ку́рице про́со\fn{hirse} сни́тся\fn{drømmer}.
        \mitem У голо́дной куме́\fn{gudmoder} хлеб на уме́.
        }
\dui{sund}
\ordsprog{For de \mwni{sunde} er alting sundt.}
         {Здоро́вому и нездоро́вое здоро́во.}
\ordsprog{Én \mw{svale} gør ingen \di{sommer}.}
         {Одна́ ла́сточка не де́лает весны́.
        \mitem Одна́ ла́сточка\fn{svale} ещё не де́лает весны́.}
\ordsprog{Intet \mw{svar} er også svar.}
         {Ти́хое молча́ние\fn{tavshed} чем не отве́т.
        \mitem Молча́ние --- знак\fn{tegn (på)} согла́сия\fn{samtykke}.}
\ordsprog{Som man råber i skoven, får man \mw{svar}.}
         {Как ау́кнется\fn{råbe "a-u"}, так и откли́кнется\fn{give lyd fra sig, svare}.
        \mitem По спро́су\fn{efterspørgsel} и отве́т.}
\ordsprog{Mangel på \mw{svigermoder} er den bedste medgift.}
         {Отсу́тствие\fn{fravær, mangel på} тёщи\fn{svigermor} лу́чшее прида́ное\fn{medgift}.}
\dui{sygdom}
\ordsprog{\mwni{Sygdom} flyver på, men kryber af.}
         {Боле́знь\fn{sygdom} вхо́дит
         пуда́ми\fn{Gammel russisk vægtenhed 1 \ru{пуд} = 16,3 kg.},
         а выхо́дит золотника́ми\fn{Gammel russisk vægtenhed 1 \ru{золотник} 4,2658 gram}.}
\dui{synde}
\dui{sove}
\ordsprog{De, som sover, \mwni{synder} ikke.}
         {Тот, кто спит, не греши́т
        \mitem Спишь --- ме́ньше греши́шь\fn{synder}.
        \mitem Бо́льше спишь, ме́ньше греши́шь.}
\ordsprog{Efter os \mw{syndfloden}.}
        {По́сле нас хоть пото́п\fn{oversvǿmmelse, syndflod}.
        \mitem По́сле нас хоть трава́\fn{græs} не расти́\fn{vokse}.
        \mitem По́сле нас хоть волк траву́ ешь.}
\ordsprog{Andre lande, andre \mw{sæder}.}
        {Что край\fn{egn, land}, то обы́чай\fn{sædvane}.
        \mitem Во вся́кой земле́\fn{land} свой обы́чай.
        \mitem Что го́род, то но́ров\fn{gammeldags: skik, sæd}
        (нрав\fn{skikke, sæder}).}
\ordsprog{Det er ikke godt for en tom \mw{sæk} at stå oprejst.}
         {Не легко́\fn{let} пусто́му\fn{tom} мешку́\fn{sæk}
         стоймя́\fn{opretstående} стоя́ть.}
\dui{søge}
\ordsprog{Hvad, du \mwni{søger}, skal du finde.}
         {
         Ищи́те, и найдёте.
         % Тот и сы́щет\fn{finder??}, кот и́щет\fn{søger??}.
         \mitem За чем пойдёшь\fn{søger}, то и найдёшь\fn{finder}.
         \mitem Кто и́щет, тот всегда́ найдёт.
         }
% \ordsprog{En times \mw{søvn} før midnat er bedre end to derefter.}
%              {Кто ра́но встаёт\fn{står op.}, тому́ Бог подаёт\fn{giver}\fn{samme betydning?}.}
% \dui{søvn}
% \ordsprog{\mwni{Søvn} er drukken mands lægedom.}{
% Бо́льше спишь, ме́ньше греши́шь.\fn{=De, som sover, synder ikke.}}
\dui{så}
\ordsprog{Hvad du \mwni{sår}, skal du og høste.}
        {Что посе́ешь\fn{sår, spreder}, то и пожнёшь\fn{høster}.
        \mitem Какова́ земля́\fn{jord}, тако́в и хлеб\fn{brød}.
        \mitem Кокова́ земля́, тако́в и плод'\fn{frugt}.}
\ordsprog{Det \mw{sår}, man altid piller ved, læges aldrig.}
              {Не сле́дует\fn{bør} береди́ть\fn{rippe op i} ста́рую ра́ну\fn{sår}.}
%slettet \ordsprog{De værste \mw{sår} er dem, der ikke bløder.}
%         {Ста́рые ра́ны зажива́ют.}
\letter{T}
%\ordsprog{Den enes vinding er den andens \mw{tab}.}
%              {Рабо́тнику алты́н, а наря́дчику рубль.}
\dui{tak}
\ordsprog{\mwni{Tak} koster ingenting.}
         {Спаси́бо вели́ко\fn{stort} сло́во (де́ло).}
\dui{tale}
\ordsprog{\mwni{Tale} er sølv, tavshed guld.}
          {Сло́во --- серебро́, молча́ние зо́лото.
        \mitem Ска́зано --- серебро́, не ска́зано --- зо́лото
        \mitem Ска́занное сло́во сере́бряное, не ска́занное --- золото́е}
\ordsprog{Ud i sød \mw{tale} ligger falskhed i dva\-le.}
         {На слова́х медо́к\fn{honning, nektar}, а на се́рдце ледо́к\fn{is.}.}
\dui{tanker}
\ordsprog{\mwni{Tanker} er toldfri.}
    {Со слов по́шлины\fn{told(afgifter)} не беру́т\fn{tager}.
        \mitem Всяк во́лен\fn{fri} ду́мать что хо́чет.}
\ordsprog{Hver ting har sin \mw{tid}.}
               {Вся́кому о́вощу\fn{grøntsag} (фру́кту) своё вре́мя.
        \mitem Знай вре́мя и ме́сто.
        % slettet \mitem Всё стои́т до поры́, до вре́мени.
        }
\ordsprog{Hver ting til sin \mw{tid}.}
               {Всему́ своё вре́мя.}
\ordsprog{Den \mw{tid}, den sorg.}
              {Отло́жим\fn{udsætter}, до поры́\fn{til senere(tid)} до вре́мени.}
\ordsprog{Kommer \mw{tid}, kommer råd.}
       {У́тро ве́чера мудрене́е\fn{klogere}.
       \mitem Поживём --- уви́дим
       \mitem Пора́ на ум наво́дит\fn{bringe, føre lede}.
        \mitem Вре́мя пока́жет\fn{vise, demonstrere}.
        \mitem Придёт беда́\fn{ulykke, nød}, ку́пишь\fn{køber} ума́.
        \mitem Вре́мя всему́ нау́чит.
        \mitem Вре́мя ра́зум\fn{fornuft, forstand} даёт.}
\dui{tid}
\ordsprog{\mwni{Tiden} læger alle sår.}
        {Вре́мя ле́чит\fn{læger, heler} все ра́ны\fn{sår}.
        \mitem Вре́мя лу́чший ле́карь\fn{læge}.
        \mitem Вре́мя вели́кий цели́тель\fn{læge}.}
\dui{tid}
\ordsprog{\mwni{Tiden} iler.}
     {День и ночь, и су́тки\fn{døgn} прочь\fn{bort(e), væk, gået}.}
\dui{tid}
\ordsprog{\mwni{Tid} er penge.}
         {Вре́мя --- де́ньги.}
\dui{tid}
\ordsprog{Ny \mwni{tider}, nye sæder.}
        {Ины́е\fn{andre} времена́, ины́е нра́вы\fn{sæder, skikke} (пе́сни\fn{sange}).
        \mitem Ино́е вре́мя, ино́е бре́мя.}
\dui{tie}
\ordsprog{Den, der \mwni{tier}, samtykker.}
              {Молча́ние --- знак\fn{tegn} согла́сия\fn{enighed}.
        \mitem До́брое молча́ние чем не отве́т?}
\dui{tie}
\ordsprog{\mwni{Tie} og tænke kan ingen krænke.}
         {Молча́нкой\fn{tavshed, tien} никого́ не оби́дишь\fn{krænke}.
         \mitem Смолча́ть никого́ не оби́дет.}
\dui{tilfredshed}
\ordsprog{\mwni{Tilfredshed} er bedre end rigdom.}
         {Дома́шный\fn{hjemlig} телёнок\fn{kalv} лу́чше замо́рской\fn{oversøisk} коро́вы\fn{ko}.}
\ordsprog{Man kan ikke gøre alle \mw{tilpas}.}
              {Всем не угоди́шь\fn{behager}.
        \mitem На весь свет не угоди́шь.
        \mitem Всем уго́длив\fn{tjenstivrig, indsmigrende, slesk}, так никому́ не приго́длив\fn{tilfredsstillet}.
        \mitem И кра́сное со́лнышко\fn{solen (diminutiv)} на всех не угожда́ет.
        \mitem На всех угожда́ть. --- самому́ в дурака́х сиде́ть.}
\ordsprog{Hver \mw{ting} må ses fra to sider.}
               {Ка́ждую вещь сле́дует рассма́тривать со всех сторо́н.}
\ordsprog{Hver \mw{ting} har to sider.}
               {Па́лка\fn{stav} о двух конца́х\fn{ender}.
        \mitem У дуби́ны\fn{kæp, stok, kølle} два конца́.}
\ordsprog{Det er umuligt at gøre to ting samtidigt.}{
Нельзя́ де́лать одновре́менно две ве́щи.
}
\ordsprog{Den ene \mw{tjeneste} er den anden værd.}
              {Услу́га\fn{tjeneste} за услу́гу.
        \mitem Долг\fn{gæld} платежём\fn{betaler} кра́сен.}
\dui{tom}
\ordsprog{\mwni{Tomme} tønder buldrer mest.}
           {Пуста́я бо́чка\fn{tønde, fad} пу́ще\fn{mere} греми́т\fn{drøner, tordner, larmer}.
        \mitem В пусто́й бо́чке и зво́ну\fn{klang} мно́го.}
\ordsprog{Е́n \mw{tosse} kan spørge om mere end ti viise
             kan besvare.}
         {Дура́к зада́ст вопро́с, а де́сять у́мных не отве́тят.
        \mitem Дура́к в во́ду ка́мень заки́нет\fn{?}, деся́теро
         у́мных не вы́тащат\fn{trække ud, slæbe ud}.}
\ordsprog{Alle gode gange \mw{tre}.}
      {Бог лю́бит тро́ицу\fn{treenighed, trio}.}
%\dui{troskab}
%\ordsprog{\mwni{Troskab} er en sjælden gæst.}{}
\dui{tryk}
\ordsprog{\mwni{Tryk} avler modtryk.}
     {Си́ла\fn{kraft, styrke} си́лу ло́мит\fn{presse, trykke ned, være ved at knække}.
        % slettet \mitem Си́ла всё ло́мит.
        }
\ordsprog{Godt \mw{træ} bærer gode \di{æbler}.}{
    Какого́ де́рево, тако́в и плод\fn{frugt}.
    \mitem По де́реву плод.
}
\ordsprog{Kroget træ kan også bære god frugt.}{
    Криво́е\fn{skæv, krum} де́рево, да я́блоки\fn{æbler} сла́дки\fn{søde}.
}
\dui{træ}
\dui{frugt}
\ordsprog{Man kender \mwni{træet} på dets frugter.}
         {
         Каково́ де́рево  тако́в и плод\fn{frugt}.
        \mitem Каково́ де́рево  таковы́ и су́чья\fn{\ru{сук}=gren}.
        \mitem По де́реву плод.
        \mitem От я́блони\fn{æbletræ} плод, а от е́ли\fn{grantræ} --- ши́шка\fn{kogle}.
%        \mitem Знать со́кола по полёту,
%               а до́брого молодца́ по похо́дке.
}
%\ordsprog{Godt \mw{træ} bærer gode æbler.}
%        {Каково́ де́рево, тако́в и плод.
%        \mitem По де́реву плод.}
\dui{træ}
\ordsprog{Når \mwni{træet} falder, vil alle samle
   spåner (brænde)/Af et fældet træ laver alle spåner.}
      {
      % Где лес\fn{skov} ру́бят\fn{hugger, fælder}, ще́пки\fn{spåner} летя́т\fn{flyver}\fn{samme betydning?}.
        В гото́вую посте́ль\fn{seng, sengetøj} всяк лечь захо́чет\fn{får lyst til}.}
\dui{træ}
\ordsprog{\mwni{Træerne} vokser ikke ind i himlen.}
        {Де́ревья до небе́с не расту́т.
        \mitem Вы́ше\fn{over} лба\fn{pande} у́ши не рaсту́т\fn{vokser}.}
\dui{tugte}
\ordsprog{Den, man \di{elsker}, \mwni{tugter} man.}
        {Кого́ люблю́, того́ и бью\fn{slår}.}
\dui{tunge}
\ordsprog{\mwni{Tungen} er et lidet lem og kommer store ting afsted.}
             {Держи́\fn{hold fast} язы́к\fn{tunge} за зуба́ми\fn{tænder}.
        \mitem Держи́ язы́к коро́че\fn{kort}.}
\ordsprog{Lejlighed gør tyve.}
{Пло́хо не клади́\fn{put, anbring}, во́ра\fn{tyv} в грех\fn{synd} не вводи́\fn{fører, leder ind i}.}
\dui{tyv}
\ordsprog{\mwni{Tyv} tror hver mand stjæler.}
           {Всяк на свой арши́н\fn{russisk mål: 71 cm} ме́рит\fn{måler, skridter af}.
        \mitem Ме́рить други́х на свой арши́н.}
%\ordsprog{Enhver er \mw{tyv} i sin næring.}
\dui{tyv}
\dui{hænge}
\dui{hatten}
\dui{løbe}
\ordsprog{Små \mwni{tyve} hænges op, store tager man hatten af for. (store lader man løbe.)}
         {Копе́ечного\fn{kopek-} (алты́нного\fn{\ru{алты́н}: tre kopek}) во́ра\fn{tyv} веша́ют\fn{hænger}, полти́нного\fn{\ru{полти́на}: halv rubel} че́ствуют\fn{ærer}.
       % \mitem Бога́тому\fn{rigmand} идти́ в суд\fn{domstol, ret} --- трын-трава́\fn{(egtl. råddent hø), det rager dem ikke.}, бе́дному\fn{fattig} --- доло́й\fn{bort, væk} голова́.
% \mitem Мелкин воро́в ве́лкют, а больнине гуляютна свобо́де.
       }
\ordsprog{En gang \mw{tyv}, altid tyv.}
         {Раз укра́л\fn{stjal}, а всё во́ром\fn{tyv} стал.}
\dui{tyvelykke}
\ordsprog{\mwni{Tyvelykke} er galgenfrist.}
         {Ско́лько во́ру не ворова́ть\fn{stjæle, rapse}, а кнута́\fn{pisk} не минова́ть\fn{undgå}.}
\dui{tålmodighed}
\ordsprog{\mwni{Tålmodighed} overvinder alt.}
   {Терпе́ние\fn{Tålmodighed} и труд\fn{arbejde, besvær} всё перетру́т\fn{overvinder}.}
\letter{U}
\dui{udholdenhed}
\ordsprog{\mwni{Udholdenhed} overvinder alt.}
         {Терпе́ние и труд всё перетру́т.}
\dui{udholdenhed}
%slettet \ordsprog{\mwni{Udholdenhed} belønnes.}
%         {Без труда́ не вы́нешь ры́бку из пруда́.}
\dui{udsætte}
\ordsprog{\mwni{Udsæt} ikke til i morgen, hvad du kan gøre i dag.}
         {Не откла́дивай\fn{henlæg, udskyd} на за́втра, что мо́жно де́лать сего́дня.
         \mitem Одно́ ны́нче\fn{i dag} лу́чше двух за́втра.}
\dui{uheld}
\ordsprog{Når \mwni{uheldet} skal være, kan
          man brække fingeren i røven på en gammel kælling.}
         {Кому́ не везёт\fn{lykkedes med, have held med}, тот в мы́ле\fn{sæbe} на гвоздь\fn{søm} наткнётся\fn{støde mod}.
        \mitem  Кому́ не везёт, тот и в творогу́\fn{knapost, skørost}
        (ква́су\fn{drik brygget på rugbrød}) на ши́ло\fn{syl} наткнётся.}
\dui{ukrudt}
\ordsprog{\mwni{Ukrudt} forgår ikke så let.}
         {Дурна́я\fn{dårligt} трава́\fn{græs} в рост\fn{vækst} идёт.
        %\mitem Худо́е спо́ро\fn{diskussion??} не изживёшь\fn{udrydde, få has på} (сживёшь) ско́ро\fn{ryge ud, trænge ud}.
        }
\dui{ulv}
 \ordsprog{\mwni{Ulven} skifter hår, men skifter ikke sind.}
         {Волк и вся́кий год линя́ет\fn{fælder, skifter ham}, а нра́ва\fn{gemyt, karakter, væsen} не меня́ет\fn{bytte ud, forandre, skifte}.}
\dui{ulv}
\ordsprog{Man må tude med de \mwni{ulve}, man er iblandt.}
         {С волка́ми жить --- по-во́лчьи выть\fn{tude, hyle}.}
\ordsprog{Den ene \mw{ulv} æder ikke den anden.}
         {Во́рон\fn{ravn} во́рону глаз гла́за не вы́клюет\fn{hakker ud}.
         % \mitem Свой своему́ понево́ле\fn{tvungent, nødvilligt} брат\fn{tager??}.
         }
% slettet \ordsprog{Almindelig \mw{ulykke} trøster.}
%         {Чужа́я беда́ --- смех.}
\ordsprog{Én \mw{ulykke} kommer sjældent alene. / Den ene ulykke har den anden ved hånden.}
         {Беда́\fn{ulykke} не хо́дит одна́.
        \mitem Одна́ беда́ не хо́дит.
%        \mitem Беда́ скоро хо́дит.
        \mitem Беда́ бе́ду ро́дит\fn{afføder, avler}.
        }
\dui{ung}
\ordsprog{\mwni{Unge} til hjælp, gamle til råd.}
         {Мо́лодость плеча́ми\fn{skuldre} кре́пче\fn{stærkere}, ста́рость голово́й.}
\dui{ungdom}
\ordsprog{\mwni{Ungdommen} raser.}
         {Вся́кая мо́лодость ре́звости\fn{løssluppenhed?} полна́.}
\dui{ungdom}
\ordsprog{\mwni{Ungdom} og skønhed forsvinder hurtigt.}
         {Не ищи́\fn{søg} красоты́\fn{skønhed}, ищи́ доброты́\fn{godhed}.
%        \mitem С лица́ не во́ду пить, уме́ла бы пироги́ печь.
}
\dui{ungdom}
\ordsprog{Hvad man i \mwni{ungdommen} nemmer,
          man ikke i alderdommen glemmer.}
         {Что в де́тстве\fn{barndom} вы́учено\fn{lært (udenad)}, в ста́рости не забыва́ется\fn{forglemmer, døser hen}.}
\dui{utak}
\ordsprog{\mwni{Utak} er verdens løn.}
         {Нет в ми́ре благода́рности\fn{taknemmelighed, påskønnelse}.
         \mitem В э́том ми́ре не заслу́жишь\fn{gjort dig fortjent til} благодарности.
        \mitem В э́том ми́ре бладода́рность не жди́\fn{forvent}.}
\letter{V}
\dui{vade}
\ordsprog{\mwni{Vad} ikke over vandet, hvis du ikke ser bunden.}
         {Не зна́я бро́ду\fn{vadested}, не су́йся\fn{stik næsen i??} в во́ду.}
\ordsprog{Det stille \mw{vand}, den dybe grund.}
        {
        В ти́хой воде́ о́муты\fn{malstrømme, hvirvler} глубоки́\fn{dybe}.
        \mitem Ти́хие во́ды глубоки́.
        % slettet \mitem В ти́хом о́муте че́рти во́дятся.
        % slettet \mitem Ти́ха(я) вода́ берега́ подмыва́ет.
        }
\ordsprog{Man skal ikke kaste snavset \mw{vand} bort, før man har det rene.}
         {Не плюй\fn{spyt} в коло́дец\fn{brønd}, пригоди́тся\fn{bliver til nytte} воды́ напи́ться\fn{at drikke}.}
\dui{vand}
\ordsprog{I rørte \mwni{vande} er godt at fiske.}
         {В му́тной\fn{uklar, grumset, plumret sløret} воде́ ры́бу лови́ть\fn{fange}\fn{samme betydning?}.}
\dui{vane}
\ordsprog{\mwni{Vanen} er det halve liv.}
         {Привы́чка\fn{vane, sædvane} --- втора́я\fn{anden} нату́ра\fn{natur} (приро́да\fn{natur}).
        % slettet \mitem Со́колу лес не в ди́во.
        }
\dui{vare}
\ordsprog{Gode \mwni{varer} roser sig selv.}
         {Хоро́ший това́р сам себя́ хва́лит\fn{roser}.}
\ordsprog{Hvad én \mw{ved}, ved ingen. Hvad to eller tre vide, det vide alle.}
         {Ска́жешь с у́ха\fn{øre} на́ ухо, узна́ют с
         угла́\fn{hjørne} на́ угол.
        \mitem Говори́шь по сове́ту, а вы́йдет по всему́ све́ту.}
\dui{vej}
\ordsprog{Tryggest er gamle \mwni{veje} og gamle venner.}
         {Но́вых друзе́й нажива́й\fn{tjen ind, erhverv dig}, а ста́рых не теря́й\fn{mist ikke, tab ikke}.}
\ordsprog{Det er bedst at køre på banet \mw{vej}.  / Den lige vej er næst, men ikke altid bedst.}
         {Где ви́ден путь прямо́й\fn{lige}, там не е́зди
         по криво́му\fn{kurvet, skæv}.}
\ordsprog{I \mw{velstand} er der mange venner.}
         {
         Мно́го друзе́й, ко́ли\fn{(gl.) hvis, når} де́нежки\fn{penge} есть
         }
\ordsprog{En god \mw{ven} er mere værd end hundrede frænder.}
         {Ста́рый друг лу́чше но́вых двух\fn{to}.}
\ordsprog{En \mw{ven} på vejen er så god som penge i pungen.}
         {Не име́й\fn{skaf dig} сто рубле́й, а име́й сто друзе́й.}
\ordsprog{Lån din \mw{ven} og afkræv din \di{fjende}.}
         {Дру́жба\fn{venskab} дру́жбой, а в карма́н\fn{lomme}  лезь\fn{stik hånden}.}
\ordsprog{Hver mands \mw{ven}, hver mands nar.}
         {Не вся́кому дру́гу верь\fn{tro på}.}
%\ordsprog{Med \mw{venlighed} (\di{lempe}) kommer man længst.}
%         {И си́ла\fn{styrke} уму́\fn{forstand} уступа́ет\fn{viger for}.}
\ordsprog{Godt \di{regnskab} giver godt \mw{venskab}.}
         {Ча́ще счёт\fn{afregning}, кре́пче\fn{stærkere} (до́льше\fn{længere (varighed)}) дру́жба.}
\dui{venskab}
\ordsprog{\mwni{Venskab} for sig, tjeneste for sig.}
        {Дру́жба\fn{venskab} дру́жбой, а слу́жба\fn{tjeneste} слу́жбой.}

\ordsprog{Noget for \di{noget}, om \mw{venskab} skal holdes.}
         {Дар\fn{gave} да́ра ждёт\fn{venter}.}
\ordsprog{Når \di{maden} er spist, er \mw{venskab} ude.}
         {Хле́ба нет, друзе́й и не быва́ло\fn{kunne forekomme, hænde}.
         \mitem Ска́терть\fn{(bord)dug} со стола́ и дру́жба\fn{venskab} сплыла́\fn{brød sammen}.
         }
\dui{ventetid}
\ordsprog{Ventetiden falder lang.}
         {Когда́ ждёжь, вре́мя идёт ме́дленно.}
\dui{verden}
%\ordsprog{\mwni{Verden} er vid.}
%         {Бе́лый\fn{hvid} свет\fn{verden} на во́лю\fn{vilje, frihed} дан\fn{given?}.}
\ordsprog{Der er en \mw{verden} uden for Verona. /
          Verden er stor og livet langt.}
         {Свет не кли́ном\fn{kile} сошёлся\fn{blev venner med}.}
\dui{verden}
\dui{vide}
\ordsprog{Hvis \di{ungdommen} \mwni{vidste}, hvis \di{alderdommen} kunne.}
         {Е́сли бы мо́лодость зна́ла, е́сли
           бы ста́рость могла́.}
%  \ordsprog{Én mands \mw{vidende} er intet vidende.}
%           {Одни́м конём всего́ по́ля не изъе́здишь.}
\dui{vin}
\ordsprog{\mwni{Vin} fryder hjertet.}
         {Вино́ се́рдце челове́ка весели́т\fn{morer, forlyster, opmuntrer}.
%        \mitem Вино́ весели́т се́рдце челове́ку.
        }
\dui{vin}
\dui{øl}
\ordsprog{Når \mwni{vinen}(øllet) går ind, går forstanden ud.}
         {Хмель\fn{humle (beruselse)} шуми́т\fn{larmer, støjer}, ум молчи́т\fn{tier}.}
%\ordsprog{Man kan ikke leve af \di{luftsteg} og %\mw{vindfrikadeller}.}
%         {Не наесться куско́м, не нате́шиться дружко́м.}
\ordsprog{In \mw{vino} veritas.}
         {и́стина в вине́.
         \mitem Вся пра́вда в вине́.
        \mitem Винцо́\fn{diminutiv af \ru{вино́}} кра́сит\fn{forskønner} се́рдце и лицо́\fn{ansigt}.}
\dui{vove}
\ordsprog{Hvo intet \mwni{vover} intet vinder.}
         {Ниче́м не ри́скуя, ничего́ не добу́дешь\fn{fanger, (frem)skaffer, tager}.
        \mitem Без риска́ жизнь пресна́\fn{fersk, usyret(om brød), fad, flov}.}
\dui{vove}
\ordsprog{Dristigt \mwni{vovet} er halvt vundet.}
         {Сме́лость\fn{dristighed, vovemod} города́ берёт\fn{indtager}.}
%\dui{vrede}
%%\ordsprog{\mwni{Vrede} er et kortvarigt raseri.}
%         {Зло́ба --- глу́пость.}
\dui{vrede}
\ordsprog{\mwni{Vrede} er kærligheds fornyelse.}
         {Ми́лые'\fn{kære, søde, rare} браня́тся\fn{skændes}, то́лько те́шатся\fn{morer sig, adspreder}.}
\dui{væg}
\ordsprog{\mwni{Væggene} har øren.}
         {Сте́ны име́ют у́ши.
        \mitem Есть у меня́ слове́чко\fn{diminutiv af \ru{сло́во}} да
          волк недале́чко (недалеко́)}
\ordsprog{Lystig \mw{vært} gør glad gæst.}
         {Како́в хозя́ни\fn{ejer, indehaver, husbond}, таковы́ и го́сти.}
\letter{Æ}
\dui{æble}
\ordsprog{Når \mwni{æblet} er modent, så falder det.}
         {Не тряси́\fn{ryst} я́блоко, пока́ зелено́\fn{grøn},
            созре́ет\fn{modner}, само́ упадёт\fn{falder}.}
\dui{æble}
\ordsprog{\mwni{Æblet} falder ikke langt fra stammen.}
         {Я́блоко от я́блони\fn{æbletræ} (я́блоньки)
             недалеко́ па́дает.
%        \mitem Ми́мо я́блоньки яблочко па́дает.
        \mitem По ма́тери до́чка.}
\dui{æg}
\ordsprog{\mwni{Ægget} vil lære hønen.}
         {Я́йца ку́рицы не у́чат.}
\ordsprog{Et råddent \mw{æg} ødelægger hele kag\-en.}
         {От одного́ по́рченого\fn{fordærvet} я́блока це́лый\fn{hel, fuld}
              воз\fn{vognlæs} загнива́ет\fn{rådner}.
        \mitem Ло́жка дёгтя испо́ртит бо́чку\fn{fad} мёду.}
\dui{ægteskab}
%\ordsprog{\mwni{Ægteskabet} er som en åleruse, de der er ude, vil ind, og de der er inde, vil ud.}
%{Оди́н жени́лся, --- свет\fn{lyset} увида́л\fn{så?}; друго́й
%          жени́лся --- с голово́ю\fn{hoved} пропа́л\fn{forsvandt?}\fn{samme betydning?}.}
% slettet \ordsprog{Ondt \mw{ægteskab} er skærsild i dette liv.}
%         {Одному́ с жено́ю ра́дость, друго́му го́ре.}
\dui{ære}
\ordsprog{Nogen får \mwni{æren}, andre besværen.}
         {Бо́льше почёта\fn{ære}, бо́льше хлопо́т\fn{besvær}.
%        \mitem Шве́цу гри́вна, закро́йчику рубль.
        }
\dui{ære}
\ordsprog{\mwni{Ære} den som æres bør.}
         {Честь тому́, кто че́сти досто́ин\fn{fortjener, værdig}.}
\dui{ærlighed}
\ordsprog{\mwni{Ærlighed} varer længst.}
         {Че́стность\fn{ærlighed} долгове́чна\fn{holdbar, solid}.
        \mitem Че́стность --- лу́чшая\fn{bedre} поли́тика\fn{politik}.
%        \mitem Пра́ведная\fn{ærligt tjent} де́нежка\fn{penge} до́ веку\fn{århundrede, tidsalder} живёт.
        %\mitem Че́стность --- лу́чшая полити́ка.
        }
\letter{Ø}
\dui{øje}
\ordsprog{Hvad \mwni{øjet} ikke ser, har hjertet ikke ondt af.}
        {Что глаза́ не ви́дят, то се́рдце
             не береди́т\fn{ripper op i}.}
\dui{øje}
\ordsprog{Fire \mwni{øjne} ser bedre end to.}
         {Ум хорошо́, два лу́чше.
         \mitem Одна́ голова́ хорошо́, а две лу́чше.
         }
\ordsprog{Øje for \mw{øje}, tand for tand.}
         {О́ко за о́ко, зуб за зуб.}
\ordsprog{Ude af \mw{øje}, ude af sind.}
         {С глаз доло́й\fn{væk, bort fra}, из се́рдца(из па́мяти\fn{hukommelse}) вон\fn{ud, væk} .
        \mitem Вон из глаз, вон из се́рдца.}
\ordsprog{Hvad man \mw{ønsker}, tror man let.}
         {Чего́ жела́ешь, о том мечта́ешь\fn{drømmer, fantaserer}.
         \mitem Ле́гко пове́рить тому́, чего́ стра́стно\fn{inderligt} жела́ешь.
        \mitem Мы ле́гко ве́рим в то, чего́ хоти́м.
        }
\dui{øre}
\ordsprog{Sultne maver har ingen \mwni{ører}.}
         {Сы́тое\fn{mæt, fed velnæret} брю́хо\fn{bug, vom} к уче́нью\fn{undervisning} глу́хо\fn{døv, tunghør}.}
\dui{øren}
\ordsprog{Små gryder har også øren.}{
У ма́леньких кувшинов\fn{kande} то́же есть у́ши.
}
\dui{ørn}
\ordsprog{\mwni{Ørne} udklægger ikke dueunger.}
        {Не роди́т\fn{føder} ве́рба\fn{pil (træ)} гру́ши\fn{pære}.
        \mitem Орёл\fn{ørn} орла́ роди́т, а сова́\fn{ugle} сову́.}
\dui{øvelse}
\ordsprog{\mwni{Øvelse} gør mester.}
        {На́вык\fn{evne, færdighed, rutine} ма́стера ста́вит\fn{anbringer, anstiller}.
        \mitem По вы́учке\fn{uddannelse, oplæren, læren udenad} ма́стера знать.
%        \mitem Чему́ учи́лся, тому́ и пригоди́лся\fn{blev til nytte(gavn)}.
        }
\letter{Å}
\dui{å}
\ordsprog{Man skal ikke gå over \mwni{åen} efter vand.}
         {Ходи́ть по́ воду за ре́чку\fn{diminutiv af \ru{река} -- flod, strøm}.
       % \mitem Е́здить в Ту́лу со свои́м самова́ром.\fn{ligner "give hvedebrød til bagerbørn"}
        }
\dui{ånd}
\ordsprog{\mwni{Ånden} er vel redebon, men kødet er skrøbeligt.}
        {Дух\fn{ånd} бодр\fn{frisk, kvik, frejdig}, плоть\fn{kød, legeme} же немо́щна\fn{kraftesløs, skrøbelig, syg, affældig}.}
\ordsprog{Af ringe \mw{årsag} stor krig.}
         {От одного́ сло́ва --- да на век\fn{århundrede, levealder} ссо́ра\fn{skænderi, uvenskab}.}
\dui{ånd}
\ordsprog{\di{Salige} er de fattige i \mwni{ånden}.}
         {Блаже́нны ни́щиe\fn{fattige} ду́хом.}
% \end{document}
%%%%%%%%%%%%%%%%%%%%%%%%%%%%%%%
%
%\printindex{dansk}{Dansk stikordsregister}
%\printindex{russisk}{Russisk stikordsregister} %\ru{Русский индекс}}
\end{document}

%sagemathcloud={"latex_command":"xelatex -interact=nonstopmode 'ordsprog-xelatex.tex'"}


